\documentclass{article}
\title{Introduction}
\author{Paulo Gugelmo Cavalheiro Dias}
\begin{document}
\maketitle

\begin{abstract}
    This work develops a model in which a good parametrization of pollution taxation policy can counter strategic incentives to cheat in their pollution level declaration from the firms. Also, this work shows that optimal policies differ if pollution persistence and the possibility of failure of the R\&D processes are taken into account. 
\end{abstract}

Several resaons motivate ourselves to study this particular case : 
\begin{itemize}
    \item Importance of the stakes of climate protection, pollution tax has been brought as a solution by many (TSE),
    \item However, this argument does not take into account strategic interactions and incentives to lie, like the Volkswagen scandal shows it
    \item Take a theoretical example : if the firm pollutes too much, but the government does not have a direct way to control the technology of the firm, there is information assymetry. 
    \item Moreover, this does not take into account the complex mechanisms of Research and Development (R\&D) processes, where some R\&D can fail, and the persistence of pollution. 
    \item In the R\&D case : efforts to make an industry greener can fail, and not manage to reach the said objectives. 
    \item In the pollution persistence case : The recent highlight of tipping points in climate change changes the way we consider climate protection.
\end{itemize}
\pagebreak

The usefulness of a pigouvian tax policy has since long been advocated to fight against negative externalities \cite{Pigou1920}.
In the contemporaneous context of environmental policy development, a form of Pigouvian tax has been the pollution tax, one example being the carbon tax [CITATION]. 
Even though difficulties in the implementation of such policy exist and are recognized, the carbon tax is nowadays known for its multiple and non-negligeable benefits [CITATION].

The European Market for pollution bond is an example of it... [CITATION].

Arguments in defense of the carbon tax rarely take into account the potential strategic interactions that arise from such a setup.
In the context of against pollution policy, several examples show us however that non-negligible incentives to cheating exist. 
In 2013, the huge Volkwswagen scandal has shown that certain policy frameworks could lead to situations where the firms had incentives to cheat on their pollution level declaration [CITATION]. 

In this situation, the study of strategic interactions seems necessary to have a clear view on the consequences of a pollution tax. 
A lot of studies with game theoretical approches exists, that focuses in different aspects of the strategic interactions between agents.
However, they do not cover... [CITATION].

Also, the complex mechanisms of Research and Development (R\&D) are less often taken into acount than what could be expected. 

Recently, the persistence of pollution has also been recognized has an important factor to take into account.
The persistence of pollution is for example closely related to the existence of tipping points in climate change, (explain a bit what it is ...) [CITATION].

...

\bibliographystyle{plain}
\bibliography{../References/bibliography}
\end{document}