\documentclass{article}

\usepackage{amsmath}
\usepackage{amssymb}
\usepackage{bm}
\usepackage[dvipsnames]{xcolor}
\usepackage{hyperref}
\usepackage{graphicx} % Required for inserting images

\title{Baseline Model - Microeconomics Project}
\author{Chambon Lionel \\ Gugelmo Cavalheiro Dias Paulo \\ Pavlovich Karina \\ Soufflet Morgane}
\date{May 1, 2024}
\hypersetup{
    colorlinks,
    citecolor=black,
    filecolor=black,
    linkcolor=black,
    urlcolor=black
}

\begin{document}

\maketitle

\begin{abstract}
    In the following paper our aim is to use game theory to determine the optimal tax the environmentally discerning government should impose on a representative firm in order to ensure the firm invests in Research and Development and upgrades its' production technology to thus minimise the pollution per unit of product produced. \\

    We start by building a classic Stackelberg game to find the optimal tax, we then proceed to building a cheap talk game with imperfect information under the assumption that the R\&D investment might not lead to successful drop in pollution per unit of product produced to investigate the optimal strategy profile for the firm and for the government.  
\end{abstract}

\pagebreak
\tableofcontents
\pagebreak

\section{Introduction}

In this work we used game theory framework to model an efficient taxation policy the government should implement, knowing that there exist strategic incentives for the firm to falsely declare the pollution per unit of good produced. We also demonstrate how the optimal taxation policy differs in case of persistent pollution and unsuccessful R\&D results. \\

We have decided to dedicate our project to solving these issues for several reasons. \\

First, the climate change is an undeniable threat to global welfare and the green tax is often highlighted one of the most efficient means of climate protection.\\

Given that in the real world the government does not have the tools to instantly verify the amount of declared pollution, several instances happened in the past when the firms that are subject to the green tax falsely declared their emitted pollution levels, one notable example of such a case is the Volkswagen scandal of 2013 (Link!). \\

Moreover, even if the government efficiently implements the optimal tax and the firm invests substantially in the R\&D aiming to transition to greener production, the pollution still might persist. The R\&D is a complex process and the success is never guaranteed. \\

In the literature we reviewed incentives to lie and the complexity of R\&D have not been taken into account when analysing the strategic interactions between the economic agents and the effectiveness of the green tax. Hence, we have decided to make it the main focus of our project.  \\


\section{Literature review}

There is a growing interest in managing the environmental pollution levels by targeting the unsustainable production. Game theory provides a good framework for the contemporary researchers to analyse the strategic interactions between the economic agents and thus to effectively trace the effects their decisions have on each other and on the environment. \\

We have found numerous research papers that explore how game theory can be applied to study the effectiveness of environmental policies (most often the green tax) imposed on the production firms by the government. \\

Vast amounts of research have been done to analyse the effectiveness of green taxes in promoting sustainable production. It has been shown that heavy taxation must be implemented to achieve a drop in pollution when the government is dealing with highly polluting enterprises (Wei Yu, Ruizhu Han, 2019). Some papers claim that the emissions tax is one of the most effective policies the government can impose to increase the environmental quality, the optimal amount of the tax being determined by the marginal environmental damage of the market share of the production firm (Dorothée Brécard, 2010). \\

Research has demonstrated that the producers must take full responsibility for the environmental, social, economic, and cultural ramifications of the undertakings they give (Veleva and Ellenbecker, 2001; Ülkü and Engau, 2021). However, it has been shown that the transition to the green production can be challenging for the firms, since it requires a major capital investment in R\&D, thus making the final goods more costly to the consumer. Hence the firms’ decision to opt for green technologies largely depends on consumers’ willingness to increase their expenses. (Conrad, 2005, Krass 2013) \\ 

That being said, the government subsidy may stimulate the firms to switch to greener production and increase the market share of green products, thus making the green goods cheaper to produce which would result in a substantial decrease in the environmental pollution. \\

On the other hand, higher subsidies might not result in a successful drop in environmental pollution, the effectiveness of the subsidy and the amount of the subsidy largely depend on the manufacturing investment, R\&D investment, and consumers' preferences. (Yantao Ling, Jing Xu, M. Ali Ülkü, 2022)


\section{Baseline Model}
We model the economy using the following assumptions: 

\begin{itemize}
    \item There exist two agents in the economy, namely the government and the producer (the firm), consumers are not included.
    \item The firms are homogeneous and therefore representable. 
\end{itemize} 

First, we model the interactions between the government and the firm using the Stackelberg game structure (a game with strategic interactions that implies perfect information, non-simultaneous actions and two players, namely the leader and the follower). The pollution in the past is not taken into account in the simplified model, but will be included in the extension. 

\subsection{Players}
We assign the role of the leader to the firm and the role of a follower to the government. The firm  “plays” first, by maximising its' profit. The government observes the action of the firm and then makes a move by setting an optimal tax depending on the firm's profit. \\

\subsection{Firm}

\subsubsection{Technology of the firm}

The firm is a monopoly that produces a single good. It's production function takes two inputs: 

\begin{itemize}
    \item R\&D investment ($r$)
    \item manufacturing investment ($m$) 
\end{itemize}

The main goal of the R\&D investment is to diminish the pollution produced by the firm per unit of product produced, while the manufacturing investment is used by the firm to produce a quantity $q$ of its' single good. \\

The pollution $g_{t}$ generated by producing one unit of good at time $t$ is defined as:

\begin{equation}
g_{t}=\frac{1}{r_{t}} \\
\end{equation}

Where $r_{t}$ is the R\&D investment done by the firm to reduce its unit pollution level $g_{t}$. \\

The manufacturing investment $m_{t}$ needed to produce a quantity $q_{t}$ of a good is defined as: 

\begin{equation}
m_{t} = q_{t}\cdot c_{t}\\
\end{equation}

Where $q_{t}$ is the quantity of good produced by the firm at time $t$ and $c_{t}$ is the cost of manufacturing one good by the firm at time $t$\\

Lastly, the firm is subject to a tax $\tau$ based on the pollution level emitted per unit produced. \\

\subsubsection{Profit of the firm}
Given those technology constraints, the firm maximises its profit: 

\begin{equation}
    \begin{split}
        &\max_{r_{t},m_{t}}{\pi_{F,t}=q_{t}\cdot\left(p_{t}-c_{t}\right)-\tau_{t}\cdot q_{t}\cdot g_{t}} \\
        &\text{subject to : } B_{t} \leq r_{t} + m_{t}
    \end{split}
\end{equation}
Where : 
\begin{itemize}
    \item $q_{t}$ is the quantity of good produced and sold at time $t$
    \item $p_{t}$ is the price at which one unit of good is sold
    \item $c_{t}$ is the cost of production of a unit of good
    \item $\tau_{t}\in\mathbb{R}_{+}$ is the amount at which one unit of good is taxed, depending on the level of pollution produced per unit, at time $t$
    \item $g_{t}\in\mathbb{R}_{+}$ is the level of pollution emitted by producing a unit of product at time $t$
    \item $B_{t}$ is the budget of the firm at time $t$
\end{itemize}

\subsection{Government}
The government maximises the social welfare, which we define as a utility function that takes the  economic activities $\text{EA}_{t}$ and the risk of environmental disaster $\text{EQ}_{t}$ as inputs: 

\begin{equation}
    u_{G,t}=\text{EA}_{t}^{\alpha}\cdot\text{EQ}_{t}^{1-\alpha}
\end{equation}
Where : 
\begin{itemize}
    \item $\text{EA}_{t}$ is the Economic Activity of society at time $t$
    \item $\text{EQ}_{t}$ is the Environmental Quality of the world at time $t$
    \item $\alpha \in \left(0,1\right)$ is the relative importance of Economic Activity compared to the Environmental Quality of the World.
\end{itemize}
This function has two main properties : 

\begin{enumerate}
    \item Positive but decreasing returns for both Economic Activity and Environmental Quality\footnote{\hyperref[sec:properties_utility_function_gov_1]{See Appendix for proof.}}.
    \item Both Economic Activity and Environmental Quality are complementary goods, as the cross-derivative is positive \footnote{\hyperref[sec:properties_utility_function_gov_2]{See Appendix for proof.}}.
\end{enumerate}

\subsection{Model solution}

\subsection{Firm's problem}
In this simplified model, we first solve the Firm's problem for one period : 
\begin{equation*}
    \begin{split}
    & \max_{r_{t},m_{t}}{\pi_{F,t}=q_{t}\cdot\left(p_{t}-c_{t}\right)-\tau_{t}\cdot q_{t}\cdot g_{t}} \\ 
    & \text{subject to : } B_{t} \leq r_{t} + m_{t}
    \end{split}
\end{equation*}
Hence the Lagrangian is: 
\begin{equation*}
    \mathcal{L} = q_{t}\cdot\left(p_{t}-c_{t}\right)-\tau_{t}\cdot q_{t}\cdot g_{t}+\lambda\left(B-r_{t}-m_{t}\right)
\end{equation*}
Expressing the $q_{t}$ as a function of $m_{t}$, and the $g_{t}$ as a function of $r_{t}$ gives: 
\begin{equation*}
    \mathcal{L} = \frac{m_{t}}{c_{t}}\cdot\left(p_{t}-c_{t}\right)-\tau_{t}\cdot \frac{m_{t}}{c_{t}}\cdot \frac{1}{r_{t}}+\lambda\left(B-r_{t}-m_{t}\right)
\end{equation*}
Applying the First Order Condition\footnote{\hyperref[sec:proof3]{See Appendix for proof.}} gets the manufacturing investment:

\begin{equation}
    m_{t}=\frac{1}{\tau_{t}}\left(r_{t}^{2}(p_{t}-c_{t})-r_{t}\cdot \tau_{t}\right)=\frac{r_{t}^{2}\left(p_{t}-c_{t}\right)}{\tau_{t}}-r_{t}
\end{equation}
This expression gives the optimal level of manufacturing investment for the firm. Since it has a negative relationship with the pollution tax $\tau_{t}$, we can conclude that, the bigger the pollution tax, the more incentives the firm has to invest in R\&D instead of simply manufacturing goods with high pollution level. 

From that, we can identify the optimal level of research $r$ for the firm \footnote{\hyperref[sec:proof4]{See Appendix for proof.}} :
\begin{equation}
    r_{t}=\sqrt{\frac{B_{t}\cdot \tau_{t}}{p_{t}-c_{t}}}
\end{equation}
Thus, the maximum profit of the firm is\footnote{\hyperref[sec:proof5]{See Appendix for proof.}}: 
\begin{equation}
    \pi^{*}_{t}=\frac{1}{c_{t}}\cdot\left({\sqrt{B_{t}\cdot (p_{t}-c_{t})}-\sqrt{\tau_{t}}}\right)^{2}
\end{equation}

\subsection{Government problem}
In this simplified model, we set the Economic Activity $\text{EA}_{t}$ equal to the profit of the firm.
The environment quality $\text{EQ}_{t}$ at time $t$ depends on the environmental pollution at time $t$. The expression for the $\text{EQ}_{t}$ is the following: 
\begin{equation}
    EQ_{t}\equiv e^{-q_{t}\cdot g_{t}}
\end{equation}
Where $q_{t}\cdot g_{t}$ is the amount of pollution produced at time $t$ \\
Thus, the maximisation program of the government is: 
\begin{equation*}
    \max_{\tau_{t}}{u_{G,t}=\pi_{t}^{\alpha}\cdot e^{-q_{t}\cdot g_{t}\cdot(1-\alpha)}}
\end{equation*}
Since the Stackelberg game implies perfect information, the government knows the allocation decisions of the firm. Therefore  the government knows the values of $r_{t}$ and $m_{t}$, and the maximisation program of the government is \footnote{\hyperref[sec:proof6]{See Appendix for proof.}}: 
\begin{equation}
    u_{G,t}=\frac{e^{\frac{1}{c_{t}}\cdot\left(1-\alpha\right)}\cdot\left(\sqrt{B(p_{t}-c_{t})}-\sqrt{\tau_{t}}\right)^{2\alpha}}{c^{\alpha}\cdot e^{\frac{1}{c_{t}}\cdot(1-\alpha)\cdot\frac{\sqrt{B_{t}(p_{t}-c_{t})}}{\sqrt{\tau_{t}}}}}
\end{equation}
To maximise this utility function, we find first derivatives with respect to $\tau_{t}$ and set them equal to zero. Solving for $\tau_{t}$ gives the expression for the optimal pollution tax $\tau_{t}^{*}$ that maximises the objective function of the government \footnote{\hyperref[sec:proof7]{See Appendix for proof.}} : 
\begin{equation}
    \tau_{t}^{*}=\frac{B_{t}\left(p_{t}-c_{t}\right)(1-\alpha)\left[{\alpha \cdot 4\cdot c_{t}+1-\alpha}-\sqrt{(1-\alpha)^{2}+\alpha\cdot8\cdot c_{t}(1-\alpha)}\right]} {\alpha^{2} \cdot 8\cdot c_{t}^{2}}
\end{equation}
We can observe that the optimal pollution tax has the following properties: 
\begin{itemize}
    \item It has a negative relationship with the manufacturing cost $c_{t}$
    \item It has a negative relationship with the preference of the government for economic activity $\alpha$
    \item It has a positive relationship with the budget of the firm cost $B_{t}$
    \item It has a positive relationship with the selling price $p_{t}$
\end{itemize}

\subsection{Interpretations and extensions}

This model allows us to say several things about the strategic interactions between the firm and the government. 

Regarding the maximum profit of the firm, we see that $\pi^{*}$ has a negative relationship with the tax.
Under the condition that $B_{t}\cdot (p_{t}-c_{t})>\tau_{t}$, this conclusion seems to be realistic and in harmony with what we would expect from a rational firm. 

Regarding the best response of the government : first of all, the tax depend on the preference of the government, translated by $\alpha$.
As defined in the utility function, $u_{G,t}=\text{EA}_{t}^{\alpha}\cdot\text{EQ}_{t}^{1-\alpha}$, $\alpha$ is the relative preference of government for Economic Activity. 
If the government suddendly has a bigger preference for economic activity, and that we have an increase of $\alpha$, then its best response would be to decrease the tax.

Secondly, we see that the tradeoff between economic activity and environmental quality is determinant in the choice of the government. 
The negative relationship between $\tau_{t}$ and $c_{t}$ and the positive relationship between $\tau_{t}$ and $B_{t}$ shows it. 
If the firm is faced with higher production costs, it is in the interest of the government to decrease the tax, even if it implies a highger pollution level. 
In the same sense, if the firm has suddenly a higher budget, its profit will be positively impacted, and the government can allow itself to tax it a bit more. 
The conclusions of this model seem coherent with real considerations. 

To extend our baseline model and make it more applicable to real life, we are now going to introduce imperfect information

\pagebreak
\section{Appendix}

\subsection{Properties of the utility function of the government}

\subsubsection{Diminishing returns of economic activity and environmental quality}
\label{sec:properties_utility_function_gov_1}
The Government utility function is :

\begin{equation*}
    u_{G,t}=\text{EA}_{t}^{\alpha}\cdot\text{EQ}_{t}^{1-\alpha}
\end{equation*}

For $\text{EA}_{t}$ the economic activity and $\text{EQ}_{t}$ the environmental quality to yields positive but diminishing returns, the first derivatives must be positive and the second derivatives must be negatives. 

We compute : 
$$\frac{\partial u_{G,t}}{\partial \text{EA}_{t}}=\alpha\cdot\text{EA}_{t}^{\alpha-1}\cdot\text{EQ}_{t}^{1-\alpha}>0$$
$$\frac{\partial u_{G,t}}{\partial \text{EQ}_{t}}=(1-\alpha)\cdot\text{EA}_{t}^{\alpha}\cdot\text{EQ}_{t}^{-\alpha}>0$$
$$\frac{\partial^2 u_{G,t}}{\partial^2 \text{EA}_{t}}=\alpha\cdot(\alpha-1)\cdot\text{EA}_{t}^{\alpha-2}\cdot\text{EQ}_{t}^{1-\alpha}<0$$
$$\frac{\partial^2 u_{G,t}}{\partial^2 \text{EQ}_{t}}=(1-\alpha)\cdot(-\alpha)\cdot\text{EA}_{t}^{\alpha-1}\cdot\text{EQ}_{t}^{-\alpha-1}<0$$

Thus, we indeed have positive but diminishing returns for both $\text{EA}_{t}$ and $\text{EQ}_{t}$.

\subsubsection{Complementarity of economic activity and environmental quality}
\label{sec:properties_utility_function_gov_2}

For $EA_{t}$ the economic activity and $EQ_{t}$ the environmental quality to be complements, the crosse-derivative has to be positive. 
We compute :  
$$\frac{\partial^2 u_{G,t}}{\partial \text{EA}_{t} \partial \text{EQ}_{t}}=\alpha\cdot(1-\alpha)\cdot\text{EA}_{t}^{\alpha-1}\cdot\text{EQ}_{t}^{-\alpha}>0$$

Thus, $EA_{t}$ and $EQ_{t}$ are complement. This means that in this model, the governement does not prefer a world without economic activity or without some environmental quality, but it prefers a mixture of them. 

\subsection{First Order Conditions}
\label{sec:proof3}
Differentiating the Lagrangian with respect to the R\&D investment $r_{t}$ gives:
$$\frac{\partial \mathcal{L}}{\partial r_{t}}=0$$
$$\Leftrightarrow\frac{\tau_{t}\cdot m_{t}}{c_{t}\cdot r^{2}_{t}}-\lambda=0$$
$$\Leftrightarrow\lambda=\frac{\tau_{t}\cdot m_{t}}{c_{t}\cdot r^{2}_{t}}$$
Differentiating the Lagrangian with respect to the manufacturing investment $m_{t}$ gives:
$$\frac{\partial \mathcal{L}}{\partial m_{t}}=0$$
$$\Leftrightarrow\frac{p_{t}-c_{t}}{c_{t}}-\frac{\tau_{t}}{c_{t}\cdot r_{t}}-\lambda=0$$
$$\Leftrightarrow\frac{p_{t}-c_{t}}{c_{t}}-\frac{\tau_{t}}{c_{t}\cdot r_{t}}-\frac{\tau_{t}\cdot m_{t}}{c_{t}\cdot r^{2}_{t}}=0$$
$$\Leftrightarrow{r^{2}_{t}}\cdot(p{t}-c_{t})-r_{t}\cdot \tau_{t}-\tau_{t}\cdot m_{t}=0$$
Hence the manufacturing investment is:
\begin{equation}
    m_{t}=\frac{1}{\tau_{t}}\left(r_{t}^{2}(p_{t}-c_{t})-r_{t}\cdot \tau_{t}\right)=\frac{r_{t}^{2}\left(p_{t}-c_{t}\right)}{\tau_{t}}-r_{t}
\end{equation}

\subsection{Optimal Level of Research}
\label{sec:proof4}
Starting with the Budget constraint of the firm:
$$B_{t}=r_{t}+m_{t}$$
Plugging into the constraint the expression for $m_{t}$ we had found earlier gives:
$$B_{t}=r_{t}+\frac{r_{t}^{2}\left(p_{t}-c_{t}\right)}{\tau_{t}}-r_{t}$$
$$\Leftrightarrow{B_{t}=\frac{r_{t}^{2}\left(p_{t}-c_{t}\right)}{\tau_{t}}}$$
Rearranging the expression with respect to the research level ${r_{t}}$ gives:
\begin{equation}
    r_{t}=\sqrt{\frac{B_{t}\cdot \tau_{t}}{p_{t}-c_{t}}}
\end{equation}

\subsection{Maximum Profit}
\label{sec:proof5}
The First Order Condition is:
$$\pi_{t}= \frac{m_{t}}{c_{t}}\cdot\left(p_{t}-c_{t}\right)-\tau_{t}\cdot \frac{m_{t}}{c_{t}}\cdot \frac{1}{r_{t}}$$
Factorising by $\frac{m_{t}}{c_{t}}$ gives:
$$\pi_{t}=\frac{m_{t}}{c_{t}}\cdot\left(p_{t}-c_{t}-\frac{\tau_{t}}{r_{t}}\right)$$
We now plug in the expression for $r_{t}$:
$$\pi_{t}=\frac{m_{t}}{c_{t}}\cdot\left(p_{t}-c_{t}-\frac{\tau_{t}}{\sqrt{\frac{B_{t}\cdot \tau_{t}}{p_{t}-c_{t}}}}\right)$$
Which is equivalent to:
$$\pi_{t}=\frac{m_{t}}{c_{t}}\cdot\left(p_{t}-c_{t}-\sqrt{\frac{\tau_{t}(p_{t}-c_{t})}{B_{t}}}\right)$$
We now plug in the expression for $m_{t}$:
$$\pi_{t}=\frac{B_{t}-\sqrt{\frac{B_{t}\cdot \tau_{t}}{p_{t}-c_{t}}}}{c_{t}}\cdot\left(p_{t}-c_{t}-\sqrt{\frac{\tau_{t}(p_{t}-c_{t})}{B_{t}}}\right)$$
Factorising by $B_{t}$ gives:
$$\pi_{t}=\frac{1}{c_{t}}\cdot\sqrt{B_{t}}\left(\sqrt{B_{t}}-\sqrt{\frac{\tau_{t}}{p_{t}-c_{t}}}\right)\cdot\left(\frac{\sqrt{B_{t}}(p_{t}-c_{t})-\sqrt{\tau_{t}(p_{t}-c_{t})}}{\sqrt{B_{t}}}\right)$$
We now simplify by $\sqrt{B_{t}}$:
$$\pi_{t}=\frac{1}{c_{t}}\cdot\left(\sqrt{B_{t}}-\sqrt{\frac{\tau_{t}}{p_{t}-c_{t}}}\right)\cdot\left(\sqrt{B_{t}}(p_{t}-c_{t})-\sqrt{\tau_{t}(p_{t}-c_{t})}\right)$$
$$\Leftrightarrow\pi_{t}=\frac{1}{c_{t}}\cdot\left(\frac{\sqrt{B_{t}\cdot (p_{t}-c_{t})}-\sqrt{\tau_{t}}}{\sqrt{p_{t}-c_{t}}}\right)\cdot\left(\sqrt{B_{t}}(p_{t}-c_{t})-\sqrt{\tau_{t}(p_{t}-c_{t})}\right)$$
Dividing by $\sqrt(p_{t}-c_{t})$:
$$\pi_{t}=\frac{1}{c_{t}}\cdot\left({\sqrt{B_{t}\cdot (p_{t}-c_{t})}-\sqrt{\tau_{t}}}\right)\cdot\left(\sqrt{B_{t}}(\sqrt{p_{t}-c_{t}})-\sqrt{\tau_{t}}\right)$$
The simplified version of the profit function is therefore
\begin{equation}
    \pi_{t}=\frac{1}{c_{t}}\cdot\left({\sqrt{B_{t}\cdot (p_{t}-c_{t})}-\sqrt{\tau_{t}}}\right)^{2}
\end{equation}
\subsection{Government maximisation problem}
\label{sec:proof6}
The government maximisation problem is:
$$\max_{\tau_{t}}{\left(\frac{1}{c_{t}}\cdot\left({\sqrt{B_{t}\cdot (p_{t}-c_{t})}-\sqrt{\tau_{t}}}\right)^{2}\right)^{\alpha}\cdot {e^{-q_{t}\cdot g_{t}}}^{(1-\alpha)}}$$
We know that:
$$q_{t}\cdot g_{t}=\frac{m_{t}}{c_{t}}\cdot\frac{1}{r_{t}}=\frac{1}{c_{t}}\left(B_{t}-\sqrt{\frac{B_{t}\cdot \tau_{t}}{p_{t}-c_{t}}}\right)\left(\frac{\sqrt{p_{t}-c_{t}}}{\sqrt{B_{t}\cdot \tau_{t}}}\right)$$
$$\Leftrightarrow{q_{t}}\cdot g_{t}=\frac{1}{c_{t}}\cdot \left(\frac{\sqrt{B_{t}\cdot(p_{t}-c_{t})}}{\sqrt{\tau_{t}}}-1\right)=\frac{1}{c_{t}}\cdot \frac{\sqrt{B_{t}\cdot(p_{t}-c_{t})}-\sqrt{\tau_{t}}}{\sqrt{\tau_{t}}}$$
We can now plug the expression for $q_{t}\cdot g_{t}$ in the government's utility function:
$${u_{G,t}=\left(\frac{1}{c_{t}}\cdot\left({\sqrt{B_{t}\cdot (p_{t}-c_{t})}-\sqrt{\tau_{t}}}\right)^{2}\right)^{\alpha}\cdot {e^{-\frac{1}{c_{t}}\cdot \frac{\sqrt{B_{t}\cdot(p_{t}-c_{t})}-\sqrt{\tau_{t}}}{\sqrt{\tau_{t}}}}}^{(1-\alpha)}}$$
The simplified utility function of the government is therefore:
$${u_{G,t}=\frac{1}{c_{t}^\alpha}\cdot\left({\sqrt{B_{t}\cdot (p_{t}-c_{t})}-\sqrt{\tau_{t}}}\right)^{2\cdot\alpha}\cdot {e^{-(1-\alpha)\cdot\frac{1}{c_{t}}\cdot \frac{\sqrt{B_{t}\cdot(p_{t}-c_{t})}-\sqrt{\tau_{t}}}{\sqrt{\tau_{t}}}}}}$$

\subsection{Optimal pollution tax}
\label{sec:proof7}

To find the optimal pollution tax, we set the derivative of the social utility function to zero : 

$$u_{G,t}=\frac{e^{\frac{1}{c_{t}}\cdot\left(1-\alpha\right)}\cdot\left(\sqrt{B(p_{t}-c_{t})}-\sqrt{\tau_{t}}\right)^{2\alpha}}{c^{\alpha}\cdot e^{\frac{1}{c_{t}}\cdot(1-\alpha)\cdot\frac{\sqrt{B_{t}(p_{t}-c_{t})}}{\sqrt{\tau_{t}}}}}$$
We see that this function is of form $f(\tau_{t})=\frac{u(\tau_{t})}{v(\tau_{t})}$, therefore its derivative is of form $f'(\tau_{t})=\frac{u'(\tau)v(\tau_{t})-v'(\tau_{t})u(\tau_{t})}{v^2(\tau_{t})}$. To have its derivative equal to zero, we only need $u'(\tau_{t})v(\tau_{t})-v'(\tau_{t})u(\tau_{t})=0$, with :
\begin{equation*}
    \begin{cases}
        u(\tau_{t})= e^{\frac{1}{c_{t}}\cdot\left(1-\alpha\right)}\cdot\left(\sqrt{B(p_{t}-c_{t})}-\sqrt{\tau_{t}}\right)^{2\alpha}\\
        v(\tau_{t})= c^{\alpha}\cdot e^{\frac{1}{c_{t}}\cdot(1-\alpha)\cdot\frac{\sqrt{B_{t}(p_{t}-c_{t})}}{\sqrt{\tau_{t}}}}
    \end{cases}
\end{equation*}
Now computing the First Order Conditions:
\begin{equation*}
    \begin{cases}
        \frac{\partial u(\tau_{t})}{\partial \tau_{t}}= e^{\frac{1}{c_{t}}\cdot\left(1-\alpha\right)}\cdot \tau_{t}^{-\frac{1}{2}}\cdot \left(-\frac{1}{2}\right)\cdot 2\alpha \left(\sqrt{B(p_{t}-c_{t})}-\sqrt{\tau_{t}}\right)^{2\alpha-1} \\
        \frac{\partial v(\tau_{t})}{\partial \tau_{t}}= c_{t}^{\alpha}\cdot e^{\frac{1}{c_{t}}\cdot(1-\alpha)\cdot\frac{\sqrt{B_{t}(p_{t}-c_{t})}}{\sqrt{\tau_{t}}}}\cdot \frac{1}{c_{t}}(1-\alpha)\cdot\sqrt{B_{t}(p_{t}-c_{t})}\cdot\left(-\frac{1}{2}\right)\cdot\tau_{t}^{-\frac{3}{2}}
    \end{cases}
\end{equation*}
$$\iff$$
\begin{equation*}
    \begin{cases}
        \frac{\partial u(\tau_{t})}{\partial \tau_{t}}= -\alpha\cdot e^{\frac{1}{c_{t}}\cdot\left(1-\alpha\right)}\cdot \tau_{t}^{-\frac{1}{2}}\cdot \left(\sqrt{B(p_{t}-c_{t})}-\sqrt{\tau_{t}}\right)^{2\alpha-1} \\
        \frac{\partial v(\tau_{t})}{\partial \tau_{t}}= -\frac{c_{t}^{\alpha-1}}{2}\cdot e^{\frac{1}{c_{t}}\cdot(1-\alpha)\cdot\frac{\sqrt{B_{t}(p_{t}-c_{t})}}{\sqrt{\tau_{t}}}}\cdot (1-\alpha)\cdot\sqrt{B_{t}(p_{t}-c_{t})}\cdot\tau_{t}^{-\frac{3}{2}}
    \end{cases}
\end{equation*}

Thus we can compute : 
\begin{equation*}
    u'(\tau_{t})v(\tau_{t})-v'(\tau_{t})u(\tau_{t})=0
\end{equation*}
$$\iff$$
\begin{equation*}
    \begin{split}
    & -\alpha\cdot e^{\frac{1}{c_{t}}\cdot\left(1-\alpha\right)}\cdot \tau_{t}^{-\frac{1}{2}}\cdot \left(\sqrt{B(p_{t}-c_{t})}-\sqrt{\tau_{t}}\right)^{2\alpha-1} \cdot  c^{\alpha}\cdot e^{\frac{1}{c_{t}}\cdot(1-\alpha)\cdot\frac{\sqrt{B_{t}(p_{t}-c_{t})}}{\sqrt{\tau_{t}}}} \\
    & +\frac{c_{t}^{\alpha-1}}{2}\cdot e^{\frac{1}{c_{t}}\cdot(1-\alpha)\cdot\frac{\sqrt{B_{t}(p_{t}-c_{t})}}{\sqrt{\tau_{t}}}}\cdot (1-\alpha)\cdot\sqrt{B_{t}(p_{t}-c_{t})}\cdot\tau_{t}^{-\frac{3}{2}}\cdot e^{\frac{1}{c_{t}}\cdot\left(1-\alpha\right)}\cdot\left(\sqrt{B(p_{t}-c_{t})}-\sqrt{\tau_{t}}\right)^{2\alpha}=0
    \end{split}
\end{equation*}
$$\iff$$
\begin{equation*}
    \begin{split}
    & \alpha\cdot e^{\frac{1}{c_{t}}\cdot\left(1-\alpha\right)}\cdot \tau_{t}^{-\frac{1}{2}}\cdot \left(\sqrt{B(p_{t}-c_{t})}-\sqrt{\tau_{t}}\right)^{2\alpha-1} \cdot  c^{\alpha} = \\
    & \frac{c_{t}^{\alpha-1}}{2}\cdot (1-\alpha)\cdot\sqrt{B_{t}(p_{t}-c_{t})}\cdot\tau_{t}^{-\frac{3}{2}}\cdot e^{\frac{1}{c_{t}}\cdot\left(1-\alpha\right)}\cdot\left(\sqrt{B(p_{t}-c_{t})}-\sqrt{\tau_{t}}\right)^{2\alpha}
    \end{split}
\end{equation*}
We simplify by $\left(\sqrt{B(p_{t}-c_{t})}-\sqrt{\tau_{t}}\right)^{2\alpha}$:
\begin{equation*}
    \begin{split}
    & \alpha\cdot e^{\frac{1}{c_{t}}\cdot\left(1-\alpha\right)}\cdot \tau_{t}^{-\frac{1}{2}}\cdot \left(\sqrt{B(p_{t}-c_{t})}-\sqrt{\tau_{t}}\right)^{-1} \cdot  c^{\alpha} = \\
    & \frac{c_{t}^{\alpha-1}}{2}\cdot (1-\alpha)\cdot\sqrt{B_{t}(p_{t}-c_{t})}\cdot\tau_{t}^{-\frac{3}{2}}\cdot e^{\frac{1}{c_{t}}\cdot\left(1-\alpha\right)}
    \end{split}
\end{equation*}
We simplify by $e^{\frac{1}{c_{t}}\cdot\left(1-\alpha\right)}$ and by $c_{t}^{\alpha}$:
\begin{equation*}
    \alpha\cdot \tau_{t}^{-\frac{1}{2}}\cdot \left(\sqrt{B(p_{t}-c_{t})}-\sqrt{\tau_{t}}\right)^{-1} = \frac{1}{2\cdot c_{t}}\cdot (1-\alpha)\cdot\sqrt{B_{t}(p_{t}-c_{t})}\cdot\tau_{t}^{-\frac{3}{2}} 
\end{equation*}
$$\iff$$
\begin{equation*}
    \alpha\cdot \left(\sqrt{B(p_{t}-c_{t})}-\sqrt{\tau_{t}}\right)^{-1} = \frac{1}{2\cdot c_{t}}\cdot (1-\alpha)\cdot\sqrt{B_{t}(p_{t}-c_{t})}\cdot\tau_{t}^{-1} 
\end{equation*}
$$\iff$$
\begin{equation*}
    \alpha = \left(\sqrt{B(p_{t}-c_{t})}-\sqrt{\tau_{t}}\right)\cdot \frac{1}{2\cdot c_{t}}\cdot (1-\alpha)\cdot\sqrt{B_{t}(p_{t}-c_{t})}\cdot\tau_{t}^{-1} 
\end{equation*}
$$\iff$$
\begin{equation*}
    \frac{\alpha \cdot 2\cdot c_{t}}{(1-\alpha)\cdot\sqrt{B_{t}(p_{t}-c_{t})}}= \frac{\left(\sqrt{B(p_{t}-c_{t})}-\sqrt{\tau_{t}}\right)}{\tau_{t}} 
\end{equation*}
$$\iff$$
\begin{equation*}
    \tau_{t}\cdot\frac{\alpha \cdot 2\cdot c_{t}}{(1-\alpha)\cdot\sqrt{B_{t}(p_{t}-c_{t})}}= \sqrt{B(p_{t}-c_{t})}-\sqrt{\tau_{t}} 
\end{equation*}
$$\iff$$
\begin{equation*}
    \tau_{t}\cdot\frac{\alpha \cdot 2\cdot c_{t}}{(1-\alpha)}+\sqrt{\tau_{t}}\cdot\sqrt{B_{t}(p_{t}-c_{t})}-B_{t}(p_{t}-c_{t})=0
\end{equation*}
$$\iff$$
\begin{equation*}
    \tau_{t}\cdot\frac{\alpha \cdot 2\cdot c_{t}}{(1-\alpha)}-B_{t}(p_{t}-c_{t})=-\sqrt{\tau_{t}}\cdot\sqrt{B_{t}(p_{t}-c_{t})}
\end{equation*}
$$\iff$$
\begin{equation*}
    \tau_{t}^{2}\cdot\left(\frac{\alpha \cdot 2\cdot c_{t}}{(1-\alpha)}\right)^{2}-2\cdot\tau_{t}\cdot B_{t}\left(p_{t}-c_{t}\right)\cdot \frac{\alpha \cdot 2\cdot c_{t}}{(1-\alpha)}+\left(B_{t}(p_{t}-c_{t})\right)^{2}={\tau_{t}}\cdot{B_{t}(p_{t}-c_{t})}
\end{equation*}
$$\iff$$
\begin{equation*}
    \tau_{t}^{2}\cdot\left(\frac{\alpha \cdot 2\cdot c_{t}}{(1-\alpha)}\right)^{2}+\tau_{t}\cdot(-2)\cdot B_{t}\left(p_{t}-c_{t}\right)\cdot \frac{\alpha \cdot 2\cdot c_{t}}{(1-\alpha)}-{\tau_{t}}\cdot{B_{t}(p_{t}-c_{t})}+\left(B_{t}(p_{t}-c_{t})\right)^{2}=0
\end{equation*}
$$\iff$$
\begin{equation*}
    \tau_{t}^{2}\cdot\left(\frac{\alpha \cdot 2\cdot c_{t}}{(1-\alpha)}\right)^{2}+\tau_{t}\cdot B_{t}\left(p_{t}-c_{t}\right)\left((-2)\cdot \frac{\alpha \cdot 2\cdot c_{t}}{(1-\alpha)}-1\right)+\left(B_{t}(p_{t}-c_{t})\right)^{2}=0
\end{equation*}
$$\iff$$
\begin{equation*}
    \tau_{t}^{2}\cdot\left(\frac{\alpha \cdot 2\cdot c_{t}}{(1-\alpha)}\right)^{2}+\tau_{t}\cdot \left(-B_{t}\right)\left(p_{t}-c_{t}\right)\left(2\cdot \frac{\alpha \cdot 2\cdot c_{t}}{(1-\alpha)}+1\right)+\left(B_{t}(p_{t}-c_{t})\right)^{2}=0
\end{equation*}
This is a standard quadratic equation $\tau_{t}^{2}\cdot a+\tau_{t}\cdot b + c =0$ that accepts two solutions : 

$$\tau^{*}=\frac{-b\pm\sqrt{b^2-4\cdot a\cdot c}}{2\cdot a}$$

With : 

\begin{itemize}
    \item $a = \left(\frac{\alpha \cdot 2\cdot c_{t}}{1-\alpha}\right)^{2}$
    \item $b = \left(-B_{t}\right)\left(p_{t}-c_{t}\right)\left(2\cdot \frac{\alpha \cdot 2\cdot c_{t}}{(1-\alpha)}+1\right)$
    \item $c = \left(B_{t}(p_{t}-c_{t})\right)^{2}$
\end{itemize}
The solution $\tau^{*}=\frac{-b+\sqrt{b^2+4\cdot a\cdot c}}{2\cdot a}$ does not make sense in the context of this problem, so we have the optimal tax : 

\begin{equation*}
    \tau_{t}^{*}=\frac{-b-\sqrt{b^2-4\cdot a\cdot c}}{2\cdot a}
\end{equation*}
$$\iff$$
\begin{equation*}
    \tau_{t}^{*}=\frac{B_{t}\left(p_{t}-c_{t}\right)\left(2\cdot \frac{\alpha \cdot 2\cdot c_{t}}{(1-\alpha)}+1\right)-\sqrt{\left[\left(-B_{t}\right)\left(p_{t}-c_{t}\right)\left(2\cdot \frac{\alpha \cdot 2\cdot c_{t}}{(1-\alpha)}+1\right)\right]^2-4\cdot \left(\frac{\alpha \cdot 2\cdot c_{t}}{1-\alpha}\right)^{2}\cdot \left(B_{t}(p_{t}-c_{t})\right)^{2}}}{2\cdot \left(\frac{\alpha \cdot 2\cdot c_{t}}{1-\alpha}\right)^{2}}
\end{equation*}
$$\iff$$
\begin{equation*}
    \tau_{t}^{*}=\frac{B_{t}\left(p_{t}-c_{t}\right)\left(2\cdot \frac{\alpha \cdot 2\cdot c_{t}}{(1-\alpha)}+1\right)-\sqrt{\left(B_{t}\left(p_{t}-c_{t}\right)\right)^{2}\left(2\cdot \frac{\alpha \cdot 2\cdot c_{t}}{(1-\alpha)}+1\right)^2-4\cdot \left(\frac{\alpha \cdot 2\cdot c_{t}}{1-\alpha}\right)^{2}\cdot \left(B_{t}(p_{t}-c_{t})\right)^{2}}}{2\cdot \left(\frac{\alpha \cdot 2\cdot c_{t}}{1-\alpha}\right)^{2}}
\end{equation*}
$$\iff$$
\begin{equation*}
    \tau_{t}^{*}=\frac{B_{t}\left(p_{t}-c_{t}\right)\left(2\cdot \frac{\alpha \cdot 2\cdot c_{t}}{(1-\alpha)}+1\right)-\sqrt{\left(B_{t}\left(p_{t}-c_{t}\right)\right)^{2}\left[\left(2\cdot \frac{\alpha \cdot 2\cdot c_{t}}{(1-\alpha)}+1\right)^2-4\cdot \left(\frac{\alpha \cdot 2\cdot c_{t}}{1-\alpha}\right)^{2}\right]}}{2\cdot \left(\frac{\alpha \cdot 2\cdot c_{t}}{1-\alpha}\right)^{2}}
\end{equation*}
$$\iff$$
\begin{equation*}
    \tau_{t}^{*}=\frac{B_{t}\left(p_{t}-c_{t}\right)\left(2\cdot \frac{\alpha \cdot 2\cdot c_{t}}{(1-\alpha)}+1\right)-B_{t}\left(p_{t}-c_{t}\right)\sqrt{\left(2\cdot \frac{\alpha \cdot 2\cdot c_{t}}{(1-\alpha)}+1\right)^2-4\cdot \left(\frac{\alpha \cdot 2\cdot c_{t}}{1-\alpha}\right)^{2}}}{2\cdot \left(\frac{\alpha \cdot 2\cdot c_{t}}{1-\alpha}\right)^{2}}
\end{equation*}
$$\iff$$
\begin{equation*}
    \tau_{t}^{*}=\frac{B_{t}\left(p_{t}-c_{t}\right)\left[\left(2\cdot \frac{\alpha \cdot 2\cdot c_{t}}{1-\alpha}+1\right)-\sqrt{\left(2\cdot \frac{\alpha \cdot 2\cdot c_{t}}{1-\alpha}+1\right)^2-4\cdot \left(\frac{\alpha \cdot 2\cdot c_{t}}{1-\alpha}\right)^{2}}\right]}{2\cdot \left(\frac{\alpha \cdot 2\cdot c_{t}}{1-\alpha}\right)^{2}}
\end{equation*}
$$\iff$$
\begin{equation*}
    \tau_{t}^{*}=\frac{B_{t}\left(p_{t}-c_{t}\right)\left[\left(\frac{\alpha \cdot 4\cdot c_{t}+1-\alpha}{1-\alpha}\right)-\sqrt{\left(\frac{\alpha \cdot 4\cdot c_{t}+1-\alpha}{1-\alpha}\right)^2-\frac{\alpha^{2} \cdot 16\cdot c_{t}^{2}}{(1-\alpha)^{2}}}\right]}{\frac{\alpha^{2} \cdot 8\cdot c_{t}^{2}}{(1-\alpha)^{2}}}
\end{equation*}
$$\iff$$
\begin{equation*}
    \tau_{t}^{*}=\frac{B_{t}\left(p_{t}-c_{t}\right)\left[\left(\frac{\alpha \cdot 4\cdot c_{t}+1-\alpha}{1-\alpha}\right)-\sqrt{\frac{1}{(1-\alpha)^{2}}\left(\left(\alpha \cdot 4\cdot c_{t}+1-\alpha\right)^2-\left(\alpha^{2} \cdot 16\cdot c_{t}^{2}\right)\right)}\right]} {\frac{\alpha^{2} \cdot 8\cdot c_{t}^{2}}{(1-\alpha)^{2}}}
\end{equation*}
$$\iff$$
\begin{equation*}
    \tau_{t}^{*}=\frac{B_{t}\left(p_{t}-c_{t}\right)\left[\left(\frac{\alpha \cdot 4\cdot c_{t}+1-\alpha}{1-\alpha}\right)-\frac{1}{(1-\alpha)}\sqrt{\alpha^{2} \cdot 16\cdot c_{t}^{2}+(1-\alpha)^{2}+2\cdot \alpha\cdot4\cdot c_{t}(1-\alpha)-\alpha^{2} \cdot 16\cdot c_{t}^{2}}\right]} {\frac{\alpha^{2} \cdot 8\cdot c_{t}^{2}}{(1-\alpha)^{2}}}
\end{equation*}
$$\iff$$
\begin{equation*}
    \tau_{t}^{*}=\frac{B_{t}\left(p_{t}-c_{t}\right)(\frac{1}{1-\alpha})\left[{\alpha \cdot 4\cdot c_{t}+1-\alpha}-\sqrt{(1-\alpha)^{2}+2\cdot \alpha\cdot4\cdot c_{t}(1-\alpha)}\right]} {\frac{\alpha^{2} \cdot 8\cdot c_{t}^{2}}{(1-\alpha)^{2}}}
\end{equation*}
$$\iff$$
\begin{equation*}\tag{9}
    \tau_{t}^{*}=\frac{B_{t}\left(p_{t}-c_{t}\right)(1-\alpha)\left[{\alpha \cdot 4\cdot c_{t}+1-\alpha}-\sqrt{(1-\alpha)^{2}+\alpha\cdot8\cdot c_{t}(1-\alpha)}\right]} {\alpha^{2} \cdot 8\cdot c_{t}^{2}}
\end{equation*}

\end{document}
