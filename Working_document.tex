\documentclass{article}

\usepackage{amsmath}
\usepackage{amssymb}
\usepackage{bm}
\usepackage[dvipsnames]{xcolor}
\usepackage{hyperref}
\usepackage{graphicx} % Required for inserting images
\usepackage{tikz}
\usepackage{amsthm}
\usetikzlibrary{positioning}
\usepackage{geometry}
\usepackage{lscape} % for landscape orientation


\title{Microeconomics Project}
\author{Chambon Lionel \\ Gugelmo Cavalheiro Dias Paulo \\ Pavlovich Karina \\ Soufflet Morgane}
\date{\today}
\hypersetup{
    colorlinks,
    citecolor=black,
    filecolor=black,
    linkcolor=black,
    urlcolor=black
}

\begin{document}

\maketitle

\begin{abstract}
    In the following paper our aim is to use game theory to determine the optimal tax the environmentally discerning government should impose on a representative firm in order to ensure the firm invests in Research and Development and upgrades its' production technology to thus minimise the pollution per unit of product produced. \\

    We start by building a classic Stackelberg game to find the optimal tax, we then proceed to building a cheap talk game with imperfect information under the assumption that the R\&D investment might not lead to successful drop in pollution per unit of product produced to investigate the optimal strategy profile for the firm and for the government.  
\end{abstract}

\pagebreak
\tableofcontents
\pagebreak

\section{Introduction}

In this work, we apply game theory framework to model an efficient taxation policy the government could implement, taking into account a strategic incentive for a firm to falsely declare the pollution per unit of good produced. We also describe the changes in the optimal taxation policy in case of persistent pollution and/or fruitless R\&D. \\

Game theory provides a beneficial framework for contemporary researchers to analyse strategic interactions between the economic agents and to effectively trace the consequences of their decisions on each other, as well as on the environment. \\

The general consensus is that the emissions tax is one of the most effective policies the government can impose to increase the environmental quality, the optimal amount of the tax being determined by the marginal environmental damage of the production firm's market share (Dorothée Brécard, 2010). In case of highly polluting enterprises, Wei Yu, Ruizhu Han show that particularly heavy taxation should be implemented (Wei Yu, Ruizhu Han, 2019). \\

Yenipazarli (2019) further examines the effectiveness of the emissions tax within a duopolistic competition framework, taking into account heterogeneous consumer preferences for greener goods.\\ 

Drake \textit{et al.} (2018) introduce asymmetric regulations across different markets in their models to analyze the presence of so-called carbon leakages, which occur when firms relocate to markets with less strict emissions regulations instead of adapting production in their domestic market.\\

Studies demonstrate that the transition to green production can be challenging for firms, since it requires a major capital investment in R\&D, making final goods more costly for the consumer. Hence the firms’ decision to opt for green technologies largely depends on consumers’ willingness to increase their expenses. (Conrad, 2005, Krass 2013) \\ 

The government subsidies may stimulate the firms to switch to greener production and increase the market share of green products, making the green goods cheaper to produce which would result in a substantial decrease in the environmental pollution.\\

That being said, Ling \textit{et al.} (2022) demonstrate that higher subsidies might not result in a successful decrease in environmental pollution, as the effectiveness of the subsidy and the amount of the subsidy largely depend on the manufacturing investment, R\&D investment, and consumers' preferences. \\

As demonstrated so far, there is substantial body of academic research that aims to investigate how to reduce environmental pollution by focusing on unsustainable production. Based on our research, we were able to identify two factors that require further analysis. \\

First, it is unlikely that the government can accurately monitor the emissions on a firm-level, thus it might be difficult to differentiate the individual-firm's emissions from aggregate levels. As a consequence, the governments might find themselves making policy decisions based on the low-quality data shared with them by the firms with outstanding credibility and spotless reputation. Several instances happened in the past when firms took advantage and falsely declared their emitted pollution levels, one notable example of such a case is the Volkswagen scandal of 2013 (Link!). \\

Second, many papers assume that an optimal level of R\&D investment \textit{perfectly} offsets or reduces production emissions. However, it is also possible that efforts to adapt production technologies might fail or do not reduce emissions. In that case, firms might achieve a sub-optimal outcome as they are forced to absorb the sunk R\&D costs.\\

In this paper we analyse the effects of informational asymmetries and frictions on the optimal tax rate based on the classic Stackelberg game framework and the "cheap talk" game. \\ 
 
\section{Baseline Model}
We model the economy using the following assumptions: 

\begin{itemize}
    \item There exist two agents in the economy, namely the government and the producer (the firm), consumers are not included.
    \item The firms are homogeneous and therefore representable. 
\end{itemize} 

First, we model the interactions between the government and the firm using the Stackelberg game structure (a game with strategic interactions that implies perfect information, non-simultaneous actions and two players, namely the leader and the follower). The pollution in the past is not taken into account in the simplified model, but will be included in the extension. 

\subsection{Players}
We assign the role of the leader to the firm and the role of a follower to the government. The firm  “plays” first, by maximising its' profit. The government observes the action of the firm and then makes a move by setting an optimal tax depending on the firm's profit. \\

\subsection{Firm}

\subsubsection{Technology of the firm}

The firm is a monopoly that produces a single good. It's production function takes two inputs: 

\begin{itemize}
    \item R\&D investment ($r$)
    \item manufacturing investment ($m$) 
\end{itemize}

The main goal of the R\&D investment is to diminish the pollution produced by the firm per unit of product produced, while the manufacturing investment is used by the firm to produce a quantity $q$ of its' single good. \\

The pollution $g_{t}$ generated by producing one unit of good at time $t$ is defined as:

\begin{equation}
g_{t}=\frac{1}{r_{t}} \\
\end{equation}

Where $r_{t}$ is the R\&D investment done by the firm to reduce its unit pollution level $g_{t}$. \\

The manufacturing investment $m_{t}$ needed to produce a quantity $q_{t}$ of a good is defined as: 

\begin{equation}
m_{t} = q_{t}\cdot c_{t}\\
\end{equation}

Where $q_{t}$ is the quantity of good produced by the firm at time $t$ and $c_{t}$ is the cost of manufacturing one good by the firm at time $t$\\

Lastly, the firm is subject to a tax $\tau$ based on the pollution level emitted per unit produced. \\

\subsubsection{Profit of the firm}
Given those technology constraints, the firm maximises its profit: 

\begin{equation}
    \begin{split}
        &\max_{r_{t},m_{t}}{\pi_{F,t}=q_{t}\cdot\left(p_{t}-c_{t}\right)-\tau_{t}\cdot q_{t}\cdot g_{t}} \\
        &\text{subject to : } B_{t} \leq r_{t} + m_{t}
    \end{split}
\end{equation}
Where : 
\begin{itemize}
    \item $q_{t}$ is the quantity of good produced and sold at time $t$
    \item $p_{t}$ is the price at which one unit of good is sold
    \item $c_{t}$ is the cost of production of a unit of good
    \item $\tau_{t}\in\mathbb{R}_{+}$ is the amount at which one unit of good is taxed, depending on the level of pollution produced per unit, at time $t$
    \item $g_{t}\in\mathbb{R}_{+}$ is the level of pollution emitted by producing a unit of product at time $t$
    \item $B_{t}$ is the budget of the firm at time $t$
\end{itemize}

\subsection{Government}
The government maximises the social welfare, which we define as a utility function that takes the  economic activities $\text{EA}{t}$ and the risk of environmental disaster $\text{EQ}{t}$ as inputs: 

\begin{equation}
    u_{G,t}=\text{EA}{t}^{\alpha}\cdot\text{EQ}{t}^{1-\alpha}
\end{equation}
Where : 
\begin{itemize}
    \item $\text{EA}_{t}$ is the Economic Activity of society at time $t$
    \item $\text{EQ}_{t}$ is the Environmental Quality of the world at time $t$
    \item $\alpha \in \left(0,1\right)$ is the relative importance of Economic Activity compared to the Environmental Quality of the World.
\end{itemize}
This function has two main properties : 

\begin{enumerate}
    \item Positive but decreasing returns for both Economic Activity and Environmental Quality\footnote{\hyperref[sec:properties_utility_function_gov_1]{See Appendix for proof.}}.
    \item Both Economic Activity and Environmental Quality are complementary goods, as the cross-derivative is positive \footnote{\hyperref[sec:properties_utility_function_gov_2]{See Appendix for proof.}}.
\end{enumerate}

\subsection{Model solution}

\subsection{Firm's problem}
In this simplified model, we first solve the Firm's problem for one period (our previous maximisation problem (3)) : 
\begin{equation*}
    \begin{split}
    & \max_{r_{t},m_{t}}{\pi_{F,t}=q_{t}\cdot\left(p_{t}-c_{t}\right)-\tau_{t}\cdot q_{t}\cdot g_{t}} \\ 
    & \text{subject to : } B_{t} \leq r_{t} + m_{t}
    \end{split}
\end{equation*}
Hence the Lagrangian is: 
\begin{equation*}
    \mathcal{L} = q_{t}\cdot\left(p_{t}-c_{t}\right)-\tau_{t}\cdot q_{t}\cdot g_{t}+\lambda\left(B-r_{t}-m_{t}\right)
\end{equation*}
Expressing the $q_{t}$ as a function of $m_{t}$, and the $g_{t}$ as a function of $r_{t}$ gives: 
\begin{equation*}
    \mathcal{L} = \frac{m_{t}}{c_{t}}\cdot\left(p_{t}-c_{t}\right)-\tau_{t}\cdot \frac{m_{t}}{c_{t}}\cdot \frac{1}{r_{t}}+\lambda\left(B-r_{t}-m_{t}\right)
\end{equation*}
Applying the First Order Condition\footnote{\hyperref[sec:proof3]{See Appendix for proof.}} gets the manufacturing investment:


\begin{equation}
    m_{t}=\frac{1}{\tau_{t}}\left(r_{t}^{2}(p_{t}-c_{t})-r_{t}\cdot \tau_{t}\right)=\frac{r_{t}^{2}\left(p_{t}-c_{t}\right)}{\tau_{t}}-r_{t}
\end{equation}
This expression gives the optimal level of manufacturing investment for the firm. Since it has a negative relationship with the pollution tax $\tau_{t}$, we can conclude that, the bigger the pollution tax, the more incentives the firm has to invest in R\&D instead of simply manufacturing goods with high pollution level. 

From that, we can identify the optimal level of research $r$ for the firm\footnote{\hyperref[sec:proof4]{See Appendix for proof.}} :
\begin{equation}
    r_{t}=\sqrt{\frac{B_{t}\cdot \tau_{t}}{p_{t}-c_{t}}}
\end{equation}
Thus, the maximum profit of the firm is \footnote[5]{\hyperref[sec:proof5]{See Appendix for proof.}} : 
\begin{equation}
    \pi_{t}=\frac{1}{c_{t}}\cdot\left({\sqrt{B_{t}\cdot (p_{t}-c_{t})}-\sqrt{\tau_{t}}}\right)^{2}
\end{equation}


\subsection{Government problem}
In this simplified model, we set the Economic Activity $\text{EA}_{t}$ equal to the profit of the firm.
The environment quality $\text{EQ}{t}$ at time $t$ depends on the environmental pollution at time $t$. The expression for the $\text{EQ}{t}$ is the following: 
\begin{equation}
    EQ_{t}\equiv e^{-q_{t}\cdot g_{t}}
\end{equation}
Where $q_{t}\cdot g_{t}$ is the amount of pollution produced at time $t$ \\
Thus, the maximisation program of the government is: 
\begin{equation*}
    \max_{\tau_{t}}{u_{G,t}=\pi_{t}^{\alpha}\cdot e^{-q_{t}\cdot g_{t}\cdot(1-\alpha)}}
\end{equation*}
Since the Stackelberg game implies perfect information, the government knows the allocation decisions of the firm. Therefore  the government knows the values of $r_{t}$ and $m_{t}$, and the maximisation program of the government is\footnote{\hyperref[sec:proof6]{See Appendix for proof.}}: 

\begin{equation}
    u_{G,t}=\frac{e^{\frac{1}{c_{t}}\cdot\left(1-\alpha\right)}\cdot\left(\sqrt{B(p_{t}-c_{t})}-\sqrt{\tau_{t}}\right)^{2\alpha}}{c^{\alpha}\cdot e^{\frac{1}{c_{t}}\cdot(1-\alpha)\cdot\frac{\sqrt{B_{t}(p_{t}-c_{t})}}{\sqrt{\tau_{t}}}}}
\end{equation}
To maximise this utility function, we find first derivatives with respect to $\tau_{t}$ and set them equal to zero. Solving for $\tau_{t}$ gives the expression for the optimal pollution tax $\tau_{t}^{*}$ that maximises the objective function of the government \footnote{\hyperref[sec:proof7]{See Appendix for proof.}} : 
\begin{equation}
    \tau_{t}^{*}=\frac{B_{t}\left(p_{t}-c_{t}\right)(1-\alpha)\left[{\alpha \cdot 4\cdot c_{t}+1-\alpha}-\sqrt{(1-\alpha)^{2}+\alpha\cdot8\cdot c_{t}(1-\alpha)}\right]} {\alpha^{2} \cdot 8\cdot c_{t}^{2}}
\end{equation}
We can observe that the optimal pollution tax has the following properties: 
\begin{itemize}
    \item It has a negative relationship with the manufacturing cost $c_{t}$
    \item It has a negative relationship with the preference of the government for economic activity $\alpha$
    \item It has a positive relationship with the budget of the firm cost $B_{t}$
    \item It has a positive relationship with the selling price $p_{t}$
\end{itemize}


\newtheorem{proposition}{Proposition}

\section{Extension}
\subsection{Motivation}
So far, we have shown that under friction-less and perfect information conditions in a Stackelberg framework, there exists a government-chosen, optimal tax level that enforces a level of R\&D spending which maximizes both firm profits and social welfare, by taking into account pollution and environmental risks. 

Now we propose a simple "cheap talk" game between the Government and the representative Firm following Crawford and Sobel (1982).

In this extension we take into account the imperfect information about the success of the technological investment. We believe that this is a suitable construct as firms could have an incentive to be dishonest about their true level of emissions in order to tempt the government to enact a lower tax thus increasing the firm profits. We also consider the case where a firm might not under-report its emissions by too large of a margin to preserve a reasonable degree of credibility. 

The following section will set up the game, present a solution, identify equilibrium conditions and discuss results.

\subsection{Game Setup}

\subsubsection{Assumptions}

As before, the players are given by a welfare-maximizing Government and a profit-maximizing Firm. 

\begin{itemize}
    \item \textit{Nature} draws a technology shock $\epsilon$ uniformly from $U \sim [0,2]$. Thus, positive and negative technology shocks are equally likely.
    \item Only the Firm observes the nature of its technology shock and its subsequent level of emissions.
    \item The Government knows the distribution of the technology shock and the lowest and highest possible level of emissions a firm can declare.
    \item Using the same procedure than in the previous section, the optimal tax $\tau$ as a function of the shock $\epsilon$ becomes:
    \item Similarly, the firm's optimal profit as a function of $\epsilon$ is:
\end{itemize} 

\subsubsection{Available Actions and Payoffs}

\begin{itemize}
\item The firm sends a message about the nature of a technology shock $\epsilon \sim [0,2]$. 
    Its payoff function is:
    \begin{equation}
    U_F = - (a - \epsilon + \delta)^2
    \end{equation}
where 
\begin{itemize}
    \item $a \in A$ is the action taken by the government afterwards to calculate the next tax $\tau$  
    \item $\delta \in [0, \epsilon]$ represents a variable that captures misreporting by the firm regarding its true $\epsilon$ that a firm might choose to convince the government to choose a lower tax level in the next period.
\end{itemize}

\item Based on its beliefs about the credibility of the firm, the government chooses $a$ when deciding the new optimal tax, where government utility is given by
\begin{equation}
    U_G = - (a - \epsilon)^2
\end{equation}
\item Optimal choices for the firm and the government, respectively, are:
    \begin{equation*}
        a_F = \epsilon - \delta
    \end{equation*} 
    \begin{equation*}
        a_G = \epsilon
    \end{equation*}
\end{itemize}
In this setup, the government wishes to perfectly anticipate the firm's true value of $\epsilon$, while the firm wishes to deviate and underreport it by a factor of $\delta$.

\subsection{Solving the Game}

\begin{proposition}
    There exists a Perfect Bayesian Equilibrium in the form of a "babbling" equilibrium in which the firm chooses a message regardless of the value of $\epsilon$. Then, the firm's best repose is to set 
    \begin{equation*}
        a(m) = \mathbb{E}[\epsilon] = 1
    \end{equation*}
    which leads to the equilibrium derived earlier.
\end{proposition}

This case on its own, however, is not particularly interesting. Therefore, we will seek to derive a two-message equilibrium to identify a "threshold of underreporting" given by $\delta$. Suppose thus that the firm can choose a message such that $a(m_1) < a(m_2)$. That is, suppose the firm faces a level of $\epsilon$ and must now decide by \textit{by how much} to underreport it, such that $\epsilon > m_2 > m_1$.

Plugging in the two messages into the firm's utility function yields the marginal benefit from choosing $a(m_2)$ over $a(m_1)$:
\begin{equation}
    \Delta(\epsilon) = - (a(m_2) - \epsilon - \delta)^2 + (a(m_1) - \epsilon - \delta)^2
\end{equation}
This is increasing in $\epsilon$.

\begin{proposition}
    There exists an $\epsilon$ such that the firm chooses $m_1$ if $\epsilon \in [0, \epsilon)$ and $m_2$ if $\epsilon \in (\epsilon^{*}, 2]$. Then, knowing the distribution of $\epsilon$ and by Bayes' rule :
    \begin{equation*}
        a(m_1) = \frac{\epsilon^*}{2}
    \end{equation*}
    \begin{equation*}
        a(m_2) = \frac{2 + \epsilon^*}{2}
    \end{equation*}
\end{proposition}
\vspace{5mm}
These conditions now enable us to solve for $\epsilon^*$ by plugging both expressions into (1).

$$\epsilon^{*} - \delta - \frac{\epsilon}{2} = \frac{2 + \epsilon}{2} - \epsilon^{*} + \delta$$
$$\Leftrightarrow\frac{2\epsilon^* - 2}{2} = 2\delta $$
\begin{equation}
    \epsilon^* = 2\delta + 1
\end{equation}

for $\delta \in [- \frac{1}{2}, 0]$ since $\epsilon \in [0,1]$.

\subsection{Discussion}

In our setting, we have identified boundaries for our underreporting parameter $\delta$. The previous result tells us that
\begin{itemize}
    \item If the firm faces the worst possible technology shock $\epsilon = 1$, then it has no incentive to report untruthfully, i.e., $\delta = 0$.
    \item If the firm faces the best possible technology shock $\epsilon = 0$, it will choose $\delta = - \frac{1}{2}$. 
\end{itemize}
\pagebreak

\section{Conclusion}
We have shown that under frictionless and perfect information conditions in a Stackelberg framework, there exists a government-chosen, optimal tax level that enforces a level of R\&D spending which maximizes both firm profits and social welfare, by taking into account pollution and environmental risks. \\

\pagebreak
\section{Appendix}

\subsection{Properties of the utility function of the government}

\subsubsection{Diminishing returns of economic activity and environmental quality}
\label{sec:properties_utility_function_gov_1}
The Government utility function is :

\begin{equation}\tag{4}
    u_{G,t}=\text{EA}_{t}^{\alpha}\cdot\text{EQ}{t}^{1-\alpha}
\end{equation}

For $\text{EA}{t}$ the economic activity and $\text{EQ}{t}$ the environmental quality to yields positive but diminishing returns, the first derivatives must be positive and the second derivatives must be negatives. 

We compute : 
$$\frac{\partial u_{G,t}}{\partial \text{EA}{t}}=\alpha\cdot\text{EA}{t}^{\alpha-1}\cdot\text{EQ}_{t}^{1-\alpha}>0$$
$$\frac{\partial u_{G,t}}{\partial \text{EQ}{t}}=(1-\alpha)\cdot\text{EA}{t}^{\alpha}\cdot\text{EQ}_{t}^{-\alpha}>0$$
$$\frac{\partial^2 u_{G,t}}{\partial^2 \text{EA}{t}}=\alpha\cdot(\alpha-1)\cdot\text{EA}{t}^{\alpha-2}\cdot\text{EQ}_{t}^{1-\alpha}<0$$
$$\frac{\partial^2 u_{G,t}}{\partial^2 \text{EQ}{t}}=(1-\alpha)\cdot(-\alpha)\cdot\text{EA}{t}^{\alpha-1}\cdot\text{EQ}_{t}^{-\alpha-1}<0$$

Thus, we indeed have positive but diminishing returns for both $\text{EA}{t}$ and $\text{EQ}{t}$.

\subsubsection{Complementarity of economic activity and environmental quality}
\label{sec:properties_utility_function_gov_2}

For $EA_{t}$ the economic activity and $EQ_{t}$ the environmental quality to be complements, the crosse-derivative has to be positive. 
We compute :  
$$\frac{\partial^2 u_{G,t}}{\partial \text{EA}{t} \partial \text{EQ}{t}}=\alpha\cdot(1-\alpha)\cdot\text{EA}{t}^{\alpha-1}\cdot\text{EQ}{t}^{-\alpha}>0$$

Thus, $EA_{t}$ and $EQ_{t}$ are complement. This means that in this model, the government does not prefer a world without economic activity or without some environmental quality, but it prefers a mixture of them. 

\subsection{First Order Conditions}
\label{sec:proof3}
Differentiating the Lagrangian with respect to the R\&D investment $r_{t}$ gives:
$$\frac{\partial \mathcal{L}}{\partial r_{t}}=0$$
$$\Leftrightarrow\frac{\tau_{t}\cdot m_{t}}{c_{t}\cdot r^{2}_{t}}-\lambda=0$$
$$\Leftrightarrow\lambda=\frac{\tau_{t}\cdot m_{t}}{c_{t}\cdot r^{2}_{t}}$$
Differentiating the Lagrangian with respect to the manufacturing investment $m_{t}$ gives:
$$\frac{\partial \mathcal{L}}{\partial m_{t}}=0$$
$$\Leftrightarrow\frac{p_{t}-c_{t}}{c_{t}}-\frac{\tau_{t}}{c_{t}\cdot r_{t}}-\lambda=0$$
$$\Leftrightarrow\frac{p_{t}-c_{t}}{c_{t}}-\frac{\tau_{t}}{c_{t}\cdot r_{t}}-\frac{\tau_{t}\cdot m_{t}}{c_{t}\cdot r^{2}_{t}}=0$$
$$\Leftrightarrow{r^{2}{t}}\cdot(p{t}-c_{t})-r_{t}\cdot \tau_{t}-\tau_{t}\cdot m_{t}=0$$
Hence the manufacturing investment is:
\begin{equation}\tag{5}
    m_{t}=\frac{1}{\tau_{t}}\left(r_{t}^{2}(p_{t}-c_{t})-r_{t}\cdot \tau_{t}\right)=\frac{r_{t}^{2}\left(p_{t}-c_{t}\right)}{\tau_{t}}-r_{t}
\end{equation}

\subsection{Optimal Level of Research}
\label{sec:proof4}
Starting with the Budget constraint of the firm:
$$B_{t}=r_{t}+m_{t}$$
Plugging into the constraint the expression for $m_{t}$ we had found earlier gives:
$$B_{t}=r_{t}+\frac{r_{t}^{2}\left(p_{t}-c_{t}\right)}{\tau_{t}}-r_{t}$$
$$\Leftrightarrow{B_{t}=\frac{r_{t}^{2}\left(p_{t}-c_{t}\right)}{\tau_{t}}}$$
Rearranging the expression with respect to the research level ${r_{t}}$ gives:
\begin{equation}\tag{6}
    r_{t}=\sqrt{\frac{B_{t}\cdot \tau_{t}}{p_{t}-c_{t}}}
\end{equation}

\subsection{Maximum Profit}
\label{sec:proof5}

The First Order Condition is:
$$\pi_{t}= \frac{m_{t}}{c_{t}}\cdot\left(p_{t}-c_{t}\right)-\tau_{t}\cdot \frac{m_{t}}{c_{t}}\cdot \frac{1}{r_{t}}$$
Factorising by $\frac{m_{t}}{c_{t}}$ gives:
$$\pi_{t}=\frac{m_{t}}{c_{t}}\cdot\left(p_{t}-c_{t}-\frac{\tau_{t}}{r_{t}}\right)$$
We now plug in the expression for $r_{t}$:
$$\pi_{t}=\frac{m_{t}}{c_{t}}\cdot\left(p_{t}-c_{t}-\frac{\tau_{t}}{\sqrt{\frac{B_{t}\cdot \tau_{t}}{p_{t}-c_{t}}}}\right)$$
Which is equivalent to:
$$\pi_{t}=\frac{m_{t}}{c_{t}}\cdot\left(p_{t}-c_{t}-\sqrt{\frac{\tau_{t}(p_{t}-c_{t})}{B_{t}}}\right)$$
We now plug in the expression for $m_{t}$:
$$\pi_{t}=\frac{B_{t}-\sqrt{\frac{B_{t}\cdot \tau_{t}}{p_{t}-c_{t}}}}{c_{t}}\cdot\left(p_{t}-c_{t}-\sqrt{\frac{\tau_{t}(p_{t}-c_{t})}{B_{t}}}\right)$$
Factorising by $B_{t}$ gives:
$$\pi_{t}=\frac{1}{c_{t}}\cdot\sqrt{B_{t}}\left(\sqrt{B_{t}}-\sqrt{\frac{\tau_{t}}{p_{t}-c_{t}}}\right)\cdot\left(\frac{\sqrt{B_{t}}(p_{t}-c_{t})-\sqrt{\tau_{t}(p_{t}-c_{t})}}{\sqrt{B_{t}}}\right)$$
We now simplify by $\sqrt{B_{t}}$:
$$\pi_{t}=\frac{1}{c_{t}}\cdot\left(\sqrt{B_{t}}-\sqrt{\frac{\tau_{t}}{p_{t}-c_{t}}}\right)\cdot\left(\sqrt{B_{t}}(p_{t}-c_{t})-\sqrt{\tau_{t}(p_{t}-c_{t})}\right)$$
$$\Leftrightarrow\pi_{t}=\frac{1}{c_{t}}\cdot\left(\frac{\sqrt{B_{t}\cdot (p_{t}-c_{t})}-\sqrt{\tau_{t}}}{\sqrt{p_{t}-c_{t}}}\right)\cdot\left(\sqrt{B_{t}}(p_{t}-c_{t})-\sqrt{\tau_{t}(p_{t}-c_{t})}\right)$$
Dividing by $\sqrt(p_{t}-c_{t})$:
$$\pi_{t}=\frac{1}{c_{t}}\cdot\left({\sqrt{B_{t}\cdot (p_{t}-c_{t})}-\sqrt{\tau_{t}}}\right)\cdot\left(\sqrt{B_{t}}(\sqrt{p_{t}-c_{t}})-\sqrt{\tau_{t}}\right)$$
The simplified version of the profit function is therefore
\begin{equation}\tag{7}
    \pi_{t}=\frac{1}{c_{t}}\cdot\left({\sqrt{B_{t}\cdot (p_{t}-c_{t})}-\sqrt{\tau_{t}}}\right)^{2}
\end{equation}
\subsection{Government maximisation problem}
\label{sec:proof6}
The government maximisation problem is:
$$\max_{\tau_{t}}{\left(\frac{1}{c_{t}}\cdot\left({\sqrt{B_{t}\cdot (p_{t}-c_{t})}-\sqrt{\tau_{t}}}\right)^{2}\right)^{\alpha}\cdot {e^{-q_{t}\cdot g_{t}}}^{(1-\alpha)}}$$
We know that:
$$q_{t}\cdot g_{t}=\frac{m_{t}}{c_{t}}\cdot\frac{1}{r_{t}}=\frac{1}{c_{t}}\left(B_{t}-\sqrt{\frac{B_{t}\cdot \tau_{t}}{p_{t}-c_{t}}}\right)\left(\frac{\sqrt{p_{t}-c_{t}}}{\sqrt{B_{t}\cdot \tau_{t}}}\right)$$
$$\Leftrightarrow{q_{t}}\cdot g_{t}=\frac{1}{c_{t}}\cdot \left(\frac{\sqrt{B_{t}\cdot(p_{t}-c_{t})}}{\sqrt{\tau_{t}}}-1\right)=\frac{1}{c_{t}}\cdot \frac{\sqrt{B_{t}\cdot(p_{t}-c_{t})}-\sqrt{\tau_{t}}}{\sqrt{\tau_{t}}}$$
We can now plug the expression for $q_{t}\cdot g_{t}$ in the government's utility function:
$${u_{G,t}=\left(\frac{1}{c_{t}}\cdot\left({\sqrt{B_{t}\cdot (p_{t}-c_{t})}-\sqrt{\tau_{t}}}\right)^{2}\right)^{\alpha}\cdot {e^{-\frac{1}{c_{t}}\cdot \frac{\sqrt{B_{t}\cdot(p_{t}-c_{t})}-\sqrt{\tau_{t}}}{\sqrt{\tau_{t}}}}}^{(1-\alpha)}}$$
The simplified utility function of the government is therefore:
$${u_{G,t}=\frac{1}{c_{t}^\alpha}\cdot\left({\sqrt{B_{t}\cdot (p_{t}-c_{t})}-\sqrt{\tau_{t}}}\right)^{2\cdot\alpha}\cdot {e^{-(1-\alpha)\cdot\frac{1}{c_{t}}\cdot \frac{\sqrt{B_{t}\cdot(p_{t}-c_{t})}-\sqrt{\tau_{t}}}{\sqrt{\tau_{t}}}}}}$$

\subsection{Optimal pollution tax}
\label{sec:proof7}

To find the optimal pollution tax, we set the derivative of the social utility function to zero : 

\begin{equation}\tag{9}
    u_{G,t}=\frac{e^{\frac{1}{c_{t}}\cdot\left(1-\alpha\right)}\cdot\left(\sqrt{B(p_{t}-c_{t})}-\sqrt{\tau_{t}}\right)^{2\alpha}}{c^{\alpha}\cdot e^{\frac{1}{c_{t}}\cdot(1-\alpha)\cdot\frac{\sqrt{B_{t}(p_{t}-c_{t})}}{\sqrt{\tau_{t}}}}}
\end{equation}
We see that this function is of form $f(\tau_{t})=\frac{u(\tau_{t})}{v(\tau_{t})}$, therefore its derivative is of form $f'(\tau_{t})=\frac{u'(\tau)v(\tau_{t})-v'(\tau_{t})u(\tau_{t})}{v^2(\tau_{t})}$. To have its derivative equal to zero, we only need $u'(\tau_{t})v(\tau_{t})-v'(\tau_{t})u(\tau_{t})=0$, with :
\begin{equation*}
    \begin{cases}
        u(\tau_{t})= e^{\frac{1}{c_{t}}\cdot\left(1-\alpha\right)}\cdot\left(\sqrt{B(p_{t}-c_{t})}-\sqrt{\tau_{t}}\right)^{2\alpha}\\
        v(\tau_{t})= c^{\alpha}\cdot e^{\frac{1}{c_{t}}\cdot(1-\alpha)\cdot\frac{\sqrt{B_{t}(p_{t}-c_{t})}}{\sqrt{\tau_{t}}}}
    \end{cases}
\end{equation*}
Now computing the First Order Conditions:
\begin{equation*}
    \begin{cases}
        \frac{\partial u(\tau_{t})}{\partial \tau_{t}}= e^{\frac{1}{c_{t}}\cdot\left(1-\alpha\right)}\cdot \tau_{t}^{-\frac{1}{2}}\cdot \left(-\frac{1}{2}\right)\cdot 2\alpha \left(\sqrt{B(p_{t}-c_{t})}-\sqrt{\tau_{t}}\right)^{2\alpha-1} \\
        \frac{\partial v(\tau_{t})}{\partial \tau_{t}}= c_{t}^{\alpha}\cdot e^{\frac{1}{c_{t}}\cdot(1-\alpha)\cdot\frac{\sqrt{B_{t}(p_{t}-c_{t})}}{\sqrt{\tau_{t}}}}\cdot \frac{1}{c_{t}}(1-\alpha)\cdot\sqrt{B_{t}(p_{t}-c_{t})}\cdot\left(-\frac{1}{2}\right)\cdot\tau_{t}^{-\frac{3}{2}}
    \end{cases}
\end{equation*}
$$\iff$$
\begin{equation*}
    \begin{cases}
        \frac{\partial u(\tau_{t})}{\partial \tau_{t}}= -\alpha\cdot e^{\frac{1}{c_{t}}\cdot\left(1-\alpha\right)}\cdot \tau_{t}^{-\frac{1}{2}}\cdot \left(\sqrt{B(p_{t}-c_{t})}-\sqrt{\tau_{t}}\right)^{2\alpha-1} \\
        \frac{\partial v(\tau_{t})}{\partial \tau_{t}}= -\frac{c_{t}^{\alpha-1}}{2}\cdot e^{\frac{1}{c_{t}}\cdot(1-\alpha)\cdot\frac{\sqrt{B_{t}(p_{t}-c_{t})}}{\sqrt{\tau_{t}}}}\cdot (1-\alpha)\cdot\sqrt{B_{t}(p_{t}-c_{t})}\cdot\tau_{t}^{-\frac{3}{2}}
    \end{cases}
\end{equation*}

Thus we can compute : 
\begin{equation*}
    u'(\tau_{t})v(\tau_{t})-v'(\tau_{t})u(\tau_{t})=0
\end{equation*}
$$\iff$$
\begin{equation*}
    \begin{split}
    & -\alpha\cdot e^{\frac{1}{c_{t}}\cdot\left(1-\alpha\right)}\cdot \tau_{t}^{-\frac{1}{2}}\cdot \left(\sqrt{B(p_{t}-c_{t})}-\sqrt{\tau_{t}}\right)^{2\alpha-1} \cdot  c^{\alpha}\cdot e^{\frac{1}{c_{t}}\cdot(1-\alpha)\cdot\frac{\sqrt{B_{t}(p_{t}-c_{t})}}{\sqrt{\tau_{t}}}} \\
    & +\frac{c_{t}^{\alpha-1}}{2}\cdot e^{\frac{1}{c_{t}}\cdot(1-\alpha)\cdot\frac{\sqrt{B_{t}(p_{t}-c_{t})}}{\sqrt{\tau_{t}}}}\cdot (1-\alpha)\cdot\sqrt{B_{t}(p_{t}-c_{t})}\cdot\tau_{t}^{-\frac{3}{2}}\cdot e^{\frac{1}{c_{t}}\cdot\left(1-\alpha\right)}\cdot\left(\sqrt{B(p_{t}-c_{t})}-\sqrt{\tau_{t}}\right)^{2\alpha}=0
    \end{split}
\end{equation*}
$$\iff$$
\begin{equation*}
    \begin{split}
    & \alpha\cdot e^{\frac{1}{c_{t}}\cdot\left(1-\alpha\right)}\cdot \tau_{t}^{-\frac{1}{2}}\cdot \left(\sqrt{B(p_{t}-c_{t})}-\sqrt{\tau_{t}}\right)^{2\alpha-1} \cdot  c^{\alpha} = \\
    & \frac{c_{t}^{\alpha-1}}{2}\cdot (1-\alpha)\cdot\sqrt{B_{t}(p_{t}-c_{t})}\cdot\tau_{t}^{-\frac{3}{2}}\cdot e^{\frac{1}{c_{t}}\cdot\left(1-\alpha\right)}\cdot\left(\sqrt{B(p_{t}-c_{t})}-\sqrt{\tau_{t}}\right)^{2\alpha}
    \end{split}
\end{equation*}
We simplify by $\left(\sqrt{B(p_{t}-c_{t})}-\sqrt{\tau_{t}}\right)^{2\alpha}$:
\begin{equation*}
    \begin{split}
    & \alpha\cdot e^{\frac{1}{c_{t}}\cdot\left(1-\alpha\right)}\cdot \tau_{t}^{-\frac{1}{2}}\cdot \left(\sqrt{B(p_{t}-c_{t})}-\sqrt{\tau_{t}}\right)^{-1} \cdot  c^{\alpha} = \\
    & \frac{c_{t}^{\alpha-1}}{2}\cdot (1-\alpha)\cdot\sqrt{B_{t}(p_{t}-c_{t})}\cdot\tau_{t}^{-\frac{3}{2}}\cdot e^{\frac{1}{c_{t}}\cdot\left(1-\alpha\right)}
    \end{split}
\end{equation*}
We divide by $e^{\frac{1}{c_{t}}\cdot\left(1-\alpha\right)}$ and by $c_{t}^{\alpha}$:
\begin{equation*}
    \alpha\cdot \tau_{t}^{-\frac{1}{2}}\cdot \left(\sqrt{B(p_{t}-c_{t})}-\sqrt{\tau_{t}}\right)^{-1} = \frac{1}{2\cdot c_{t}}\cdot (1-\alpha)\cdot\sqrt{B_{t}(p_{t}-c_{t})}\cdot\tau_{t}^{-\frac{3}{2}} 
\end{equation*}
$$\iff$$
\begin{equation*}
    \alpha\cdot \left(\sqrt{B(p_{t}-c_{t})}-\sqrt{\tau_{t}}\right)^{-1} = \frac{1}{2\cdot c_{t}}\cdot (1-\alpha)\cdot\sqrt{B_{t}(p_{t}-c_{t})}\cdot\tau_{t}^{-1} 
\end{equation*}
Rearranging:
\begin{equation*}
    \tau_{t}\cdot\frac{\alpha \cdot 2\cdot c_{t}}{(1-\alpha)\cdot\sqrt{B_{t}(p_{t}-c_{t})}}= \sqrt{B(p_{t}-c_{t})}-\sqrt{\tau_{t}} 
\end{equation*}
Multiplying both sides by $\sqrt{B_{t}(p_{t}-c_{t})}$ and rearranging:
\begin{equation*}
    \tau_{t}\cdot\frac{\alpha \cdot 2\cdot c_{t}}{(1-\alpha)}=B_{t}(p_{t}-c_{t})-\sqrt{\tau_{t}}\cdot\sqrt{B_{t}(p_{t}-c_{t})} 
\end{equation*}
$$\iff$$
\begin{equation*}
    \tau_{t}\cdot\frac{\alpha \cdot 2\cdot c_{t}}{(1-\alpha)}-B_{t}(p_{t}-c_{t})=-\sqrt{\tau_{t}}\cdot\sqrt{B_{t}(p_{t}-c_{t})}
\end{equation*}
We now square both sides of the expression:
\begin{equation*}
    \tau_{t}^{2}\cdot\left(\frac{\alpha \cdot 2\cdot c_{t}}{(1-\alpha)}\right)^{2}-2\cdot\tau_{t}\cdot B_{t}\left(p_{t}-c_{t}\right)\cdot \frac{\alpha \cdot 2\cdot c_{t}}{(1-\alpha)}+\left(B_{t}(p_{t}-c_{t})\right)^{2}={\tau_{t}}\cdot{B_{t}(p_{t}-c_{t})}
\end{equation*}
$$\iff$$
\begin{equation*}
    \tau_{t}^{2}\cdot\left(\frac{\alpha \cdot 2\cdot c_{t}}{(1-\alpha)}\right)^{2}-2\cdot\tau_{t}\cdot B_{t}\left(p_{t}-c_{t}\right)\cdot \frac{\alpha \cdot 2\cdot c_{t}}{(1-\alpha)}-{\tau_{t}}\cdot{B_{t}(p_{t}-c_{t})}+\left(B_{t}(p_{t}-c_{t})\right)^{2}=0
\end{equation*}
Factorizing by $-{\tau_{t}}\cdot{B_{t}(p_{t}-c_{t})}$ gives:
\begin{equation*}
    \tau_{t}^{2}\cdot\left(\frac{\alpha \cdot 2\cdot c_{t}}{(1-\alpha)}\right)^{2}+\tau_{t}\cdot \left(-B_{t}\right)\left(p_{t}-c_{t}\right)\left(2\cdot \frac{\alpha \cdot 2\cdot c_{t}}{(1-\alpha)}+1\right)+\left(B_{t}(p_{t}-c_{t})\right)^{2}=0
\end{equation*}
This is a standard quadratic equation $\tau_{t}^{2}\cdot a+\tau_{t}\cdot b + c =0$ that accepts two solutions : 

$$\tau^{*}=\frac{-b\pm\sqrt{b^2-4\cdot a\cdot c}}{2\cdot a}$$

With : 

\begin{itemize}
    \item $a = \left(\frac{\alpha \cdot 2\cdot c_{t}}{1-\alpha}\right)^{2}$
    \item $b = \left(-B_{t}\right)\left(p_{t}-c_{t}\right)\left(2\cdot \frac{\alpha \cdot 2\cdot c_{t}}{(1-\alpha)}+1\right)$
    \item $c = \left(B_{t}(p_{t}-c_{t})\right)^{2}$
\end{itemize}
The solution $\tau^{*}=\frac{-b+\sqrt{b^2+4\cdot a\cdot c}}{2\cdot a}$ does not make sense in the context of this problem, so we have the optimal tax : 

\begin{equation*}
    \tau_{t}^{*}=\frac{-b-\sqrt{b^2-4\cdot a\cdot c}}{2\cdot a}
\end{equation*}
$$\iff$$
\begin{equation*}
    \tau_{t}^{*}=\frac{B_{t}\left(p_{t}-c_{t}\right)\left(2\cdot \frac{\alpha \cdot 2\cdot c_{t}}{(1-\alpha)}+1\right)-\sqrt{\left[\left(-B_{t}\right)\left(p_{t}-c_{t}\right)\left(2\cdot \frac{\alpha \cdot 2\cdot c_{t}}{(1-\alpha)}+1\right)\right]^2-4\cdot \left(\frac{\alpha \cdot 2\cdot c_{t}}{1-\alpha}\right)^{2}\cdot \left(B_{t}(p_{t}-c_{t})\right)^{2}}}{2\cdot \left(\frac{\alpha \cdot 2\cdot c_{t}}{1-\alpha}\right)^{2}}
\end{equation*}
$$\iff$$
\begin{equation*}
    \tau_{t}^{*}=\frac{B_{t}\left(p_{t}-c_{t}\right)\left(2\cdot \frac{\alpha \cdot 2\cdot c_{t}}{(1-\alpha)}+1\right)-\sqrt{\left(B_{t}\left(p_{t}-c_{t}\right)\right)^{2}\left(2\cdot \frac{\alpha \cdot 2\cdot c_{t}}{(1-\alpha)}+1\right)^2-4\cdot \left(\frac{\alpha \cdot 2\cdot c_{t}}{1-\alpha}\right)^{2}\cdot \left(B_{t}(p_{t}-c_{t})\right)^{2}}}{2\cdot \left(\frac{\alpha \cdot 2\cdot c_{t}}{1-\alpha}\right)^{2}}
\end{equation*}
$$\iff$$
\begin{equation*}
    \tau_{t}^{*}=\frac{B_{t}\left(p_{t}-c_{t}\right)\left(2\cdot \frac{\alpha \cdot 2\cdot c_{t}}{(1-\alpha)}+1\right)-\sqrt{\left(B_{t}\left(p_{t}-c_{t}\right)\right)^{2}\left[\left(2\cdot \frac{\alpha \cdot 2\cdot c_{t}}{(1-\alpha)}+1\right)^2-4\cdot \left(\frac{\alpha \cdot 2\cdot c_{t}}{1-\alpha}\right)^{2}\right]}}{2\cdot \left(\frac{\alpha \cdot 2\cdot c_{t}}{1-\alpha}\right)^{2}}
\end{equation*}
$$\iff$$
\begin{equation*}
    \tau_{t}^{*}=\frac{B_{t}\left(p_{t}-c_{t}\right)\left(2\cdot \frac{\alpha \cdot 2\cdot c_{t}}{(1-\alpha)}+1\right)-B_{t}\left(p_{t}-c_{t}\right)\sqrt{\left(2\cdot \frac{\alpha \cdot 2\cdot c_{t}}{(1-\alpha)}+1\right)^2-4\cdot \left(\frac{\alpha \cdot 2\cdot c_{t}}{1-\alpha}\right)^{2}}}{2\cdot \left(\frac{\alpha \cdot 2\cdot c_{t}}{1-\alpha}\right)^{2}}
\end{equation*}
$$\iff$$
\begin{equation*}
    \tau_{t}^{*}=\frac{B_{t}\left(p_{t}-c_{t}\right)\left[\left(2\cdot \frac{\alpha \cdot 2\cdot c_{t}}{1-\alpha}+1\right)-\sqrt{\left(2\cdot \frac{\alpha \cdot 2\cdot c_{t}}{1-\alpha}+1\right)^2-4\cdot \left(\frac{\alpha \cdot 2\cdot c_{t}}{1-\alpha}\right)^{2}}\right]}{2\cdot \left(\frac{\alpha \cdot 2\cdot c_{t}}{1-\alpha}\right)^{2}}
\end{equation*}
$$\iff$$
\begin{equation*}
    \tau_{t}^{*}=\frac{B_{t}\left(p_{t}-c_{t}\right)\left[\left(\frac{\alpha \cdot 4\cdot c_{t}+1-\alpha}{1-\alpha}\right)-\sqrt{\left(\frac{\alpha \cdot 4\cdot c_{t}+1-\alpha}{1-\alpha}\right)^2-\frac{\alpha^{2} \cdot 16\cdot c_{t}^{2}}{(1-\alpha)^{2}}}\right]}{\frac{\alpha^{2} \cdot 8\cdot c_{t}^{2}}{(1-\alpha)^{2}}}
\end{equation*}
$$\iff$$
\begin{equation*}
    \tau_{t}^{*}=\frac{B_{t}\left(p_{t}-c_{t}\right)\left[\left(\frac{\alpha \cdot 4\cdot c_{t}+1-\alpha}{1-\alpha}\right)-\sqrt{\frac{1}{(1-\alpha)^{2}}\left(\left(\alpha \cdot 4\cdot c_{t}+1-\alpha\right)^2-\left(\alpha^{2} \cdot 16\cdot c_{t}^{2}\right)\right)}\right]} {\frac{\alpha^{2} \cdot 8\cdot c_{t}^{2}}{(1-\alpha)^{2}}}
\end{equation*}
$$\iff$$
\begin{equation*}
    \tau_{t}^{*}=\frac{B_{t}\left(p_{t}-c_{t}\right)\left[\left(\frac{\alpha \cdot 4\cdot c_{t}+1-\alpha}{1-\alpha}\right)-\frac{1}{(1-\alpha)}\sqrt{\alpha^{2} \cdot 16\cdot c_{t}^{2}+(1-\alpha)^{2}+2\cdot \alpha\cdot4\cdot c_{t}(1-\alpha)-\alpha^{2} \cdot 16\cdot c_{t}^{2}}\right]} {\frac{\alpha^{2} \cdot 8\cdot c_{t}^{2}}{(1-\alpha)^{2}}}
\end{equation*}
$$\iff$$
\begin{equation*}
    \tau_{t}^{*}=\frac{B_{t}\left(p_{t}-c_{t}\right)(\frac{1}{1-\alpha})\left[{\alpha \cdot 4\cdot c_{t}+1-\alpha}-\sqrt{(1-\alpha)^{2}+2\cdot \alpha\cdot4\cdot c_{t}(1-\alpha)}\right]} {\frac{\alpha^{2} \cdot 8\cdot c_{t}^{2}}{(1-\alpha)^{2}}}
\end{equation*}
$$\iff$$
\begin{equation*}\tag{10}
    \tau_{t}^{*}=\frac{B_{t}\left(p_{t}-c_{t}\right)(1-\alpha)\left[{\alpha \cdot 4\cdot c_{t}+1-\alpha}-\sqrt{(1-\alpha)^{2}+\alpha\cdot8\cdot c_{t}(1-\alpha)}\right]} {\alpha^{2} \cdot 8\cdot c_{t}^{2}}
\end{equation*}

\end{document}