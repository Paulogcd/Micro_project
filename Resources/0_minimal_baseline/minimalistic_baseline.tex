\documentclass{article}
\title{Default Title}
\author{Paulo Gugelmo Cavalheiro Dias}
\begin{document}
\maketitle

\section{Baseline model}

\subsection{Agents}

In this very first model, we consider two agents : the government and the firm. 

\subsection{The government}

In a very minimalistic model, we can assume that the government retires utility from the environmental quality. 
We could thus advance no functional form of the utility function of the government, and argue that its utility can take two values : a superior value in a non polluted environment, and an inferior value in a polluted environment.

The strategies of the government are also limited : it can only choose if it implements a pollution tax or not.

\subsection{The firm}

In this baseline model, the firm only retires utility from its profit.
It has only two possible strategies : do nothing or adapt its production technology so that it pollutes less. \\
Changing its production process imply an investment $I$, but allows to avoid being taxed in case the government decides to implement the tax. \\
Not changing its production process allows the firm to not invest.
However, if the government decides to implement a pollution tax, the profit of the firm will be diminished by the tax such that : $\pi-T$.

\subsection{Strategic interations}

The strategic interactions between those two described agents could thus be summarized in the following table, with the firm as the player one, and government as player two. 

\begin{center}
    \begin{tabular}{ |c|c|c| } 
     \hline
     & Pollution tax & No pollution tax \\ 
     Change of production process & $\left(\pi-I,\overline{u_{G}}\right)$ & $\left(\pi-I,\overline{u_{G}}\right)$  \\ 
     No change of production process & $\left(\pi-T,\underline{u_{G}}\right)$ & $\left(\pi,\underline{u_{G}}\right)$ \\ 
     \hline
    \end{tabular}
    \end{center}

\subsection{Solution and discussion}

In this minimalistic setup, it is strictly more profitable for the government if the firm change its production process. 
The choice of changing or not the production process of the firm will however depend on two values : the tax $T$ and the investment $I$.
Thus, the optimal strategy for the government is to implement a tax $T$ strictly superior to the investment of the firm to change the production process. 

\end{document}