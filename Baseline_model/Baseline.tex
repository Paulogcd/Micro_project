\documentclass{article}
\usepackage{amsmath}
\usepackage{amssymb}
\usepackage{bm}
\usepackage[dvipsnames]{xcolor}

\title{Microeconomics Project}
\author{Chambon Lionel \\ Gugelmo Cavalheiro Dias Paulo \\ Pavlovich Karina \\ Soufflet Morgane}
\date{\today}


\begin{document}

\maketitle

\section{Baseline model}

This model does not use strategic interactions, and takes place within a perfect information framework. 

\section{Players}

This model has two agents : the Firm and the Governement. 

\subsection{Firm}

\subsubsection{Technology of the firm}

The firm is a monopoly, that has two possible inputs : R\&D investment ($r$) to diminish its pollution per unit and manufacturing investment ($m$) to produce a quantity $q$ of its single good. The firm is also taxed on the pollution level they emit by a tax $\tau$ implemented by the government.

\begin{equation}
    \begin{cases}
        g_{t}=\frac{1}{r_{t}} \\
        m_{t} = q_{t}\cdot c_{t}
    \end{cases}
\end{equation}

Where : 
\begin{itemize}
    \item $g_{t}$ is the pollution generated by the production of one unit of good by the firm at time $t$
    \item $r_{t}$ is the R\&D investment done by the firm to reduce its unit pollution level $g_{t}$
    \item $m_{t}$ is the manufacturing investment of the firm, that is needed to produce a certain quantity of good $q_{t}$
    \item $q_{t}$ is the quantity of good produced by the firm at time $t$
    \item $c_{t}$ is the cost of manufacturing one good by the firm at time $t$
\end{itemize}

\subsubsection{Profit of the firm}

Given those technology constraints, the firm maximises its profit such that : 

\begin{equation}
    \begin{split}
        &\max_{r_{t},m_{t}}{\pi_{F,t}=q_{t}\cdot\left(p_{t}-c_{t}\right)-\tau_{t}\cdot q_{t}\cdot g_{t}} \\
        &\text{subject to : } B_{t} \leq r_{t} + m_{t}
    \end{split}
\end{equation}

Where : 
\begin{itemize}
    \item $q_{t}$ is the quantity of good produced and sold at time $t$
    \item $p_{t}$ is the price at which one unit of good is sold
    \item $c_{t}$ is the cost of production of unit of good
    \item $\tau_{t}\in\left(0,1\right)$ the share at which the one unit of good is taxed, depending on the level of pollution produced per unit, at time $_{t}$
    \item $g_{t}\in\mathbb{R}_{+}$ is the level of pollution created by the production of one unit by the firm at time $t$
    \item $B_{t}$ is the budget of the firm at time $t$
\end{itemize}

\subsection{Government}

The goal of the government is to maximise the social welfare function. The social welfare function here is defined as a utility function taking into account economic activities and the risk of environmental disaster. We define it as : 

\begin{equation}
    u_{G,t}=\text{EA}_{t}^{\alpha}\cdot\text{EQ}_{t}^{1-\alpha}
\end{equation}

Where : 
\begin{itemize}
    \item $\text{EA}_{t}$ is the Economic Activity of society at time $t$
    \item $EQ_{t}$ is the Environmental Quality of the world at time $t$
    \item $\alpha \in \left(0,1\right)$ is the relative importance of Economic Activity compared to the Environmental Quality of the World.
\end{itemize}

This function has two main properties : 

\begin{enumerate}
    \item It has positive but decreasing returns for both Economic Activity and Environmental Quality : 

$$\frac{\partial u_{G,t}}{\partial \text{EA}_{t}}=\alpha\cdot\text{EA}_{t}^{\alpha-1}\cdot\text{EQ}_{t}^{1-\alpha}>0$$
$$\frac{\partial u_{G,t}}{\partial \text{EQ}_{t}}=(1-\alpha)\cdot\text{EA}_{t}^{\alpha}\cdot\text{EQ}_{t}^{-\alpha}>0$$

$$\frac{\partial^2 u_{G,t}}{\partial^2 \text{EA}_{t}}=\alpha\cdot(\alpha-1)\cdot\text{EA}_{t}^{\alpha-2}\cdot\text{EQ}_{t}^{1-\alpha}<0$$
$$\frac{\partial^2 u_{G,t}}{\partial^2 \text{EQ}_{t}}=(1-\alpha)\cdot(-\alpha)\cdot\text{EA}_{t}^{\alpha-1}\cdot\text{EQ}_{t}^{-\alpha-1}<0$$

    \item Both Economic Activity and Environmental Quality are complementary goods, as the cross-derivative is positive :
    
$$\frac{\partial^2 u_{G,t}}{\partial \text{EA}_{t} \partial \text{EQ}_{t}}=\alpha\cdot(1-\alpha)\cdot\text{EA}_{t}^{\alpha-1}\cdot\text{EQ}_{t}^{-\alpha}>0$$
\end{enumerate}

\section{Environment Quality}

The environment quality $EQ_{t}$ at time $t$ depends of the pollution in nature at time $t$. We can define in a simple, one period context : 

\begin{equation}
    EQ_{t}:=e^{-q_{t}\cdot g_{t}}
\end{equation}

Where :
\begin{itemize}
    \item $q_{t}\cdot g_{t}$ is the quantity of pollution produced at time $t$
\end{itemize}

Note here that we don't take into account past pollution in this simplified model. This will be taken into account in the next sections. 

\section{Model solution}

\subsection{Firm's problem}

In this simplified model, we first solve the Firm's problem for one period : 

\begin{equation*}
    \begin{split}
    & \max_{r_{t},m_{t}}{\pi_{F,t}=q_{t}\cdot\left(p_{t}-c_{t}\right)-\tau_{t}\cdot q_{t}\cdot g_{t}} \\ 
    & \text{subject to : } B_{t} \leq r_{t} + m_{t}
    \end{split}
\end{equation*}

We can thus write the Lagrangien such that : 

\begin{equation*}
    \mathcal{L} = q_{t}\cdot\left(p_{t}-c_{t}\right)-\tau_{t}\cdot q_{t}\cdot g_{t}+\lambda\left(B-r_{t}-m_{t}\right)
\end{equation*}

And if we rewrite it with the extended expression of $q_{t}$ and $m_{t}$, we get : 

\begin{equation*}
    \mathcal{L} = \frac{m_{t}}{c_{t}}\cdot\left(p_{t}-c_{t}\right)-\tau_{t}\cdot \frac{m_{t}}{c_{t}}\cdot \frac{1}{r_{t}}+\lambda\left(B-r_{t}-m_{t}\right)
\end{equation*}

We solve the First Order Conditions.

If we differentiate the Lagrangien with respect to the R\&D investment $r_{t}$ :

$$\frac{\partial \mathcal{L}}{\partial r_{t}}=0$$
$$\iff$$
$$\frac{\tau_{t}\cdot m_{t}}{c_{t}\cdot r^{2}_{t}}-\lambda=0$$
$$\iff$$
$$\lambda=\frac{\tau_{t}\cdot m_{t}}{c_{t}\cdot r^{2}_{t}}$$

If we differentiate the Lagrangien with respect to the manufacturing investment $m_{t}$ :

$$\frac{\partial \mathcal{L}}{\partial m_{t}}=0$$
$$\iff$$
$$\frac{p_{t}-c_{t}}{c_{t}}-\frac{\tau_{t}}{c_{t}\cdot r_{t}}-\lambda=0$$
$$\iff$$
$$\frac{p_{t}-c_{t}}{c_{t}}-\frac{\tau_{t}}{c_{t}\cdot r_{t}}-\frac{\tau_{t}\cdot m_{t}}{c_{t}\cdot r^{2}_{t}}=0$$
$$\iff$$
$$r^{2}_{t}\cdot(p_{t}-c_{t})-r_{t}\cdot \tau_{t}-\tau_{t}\cdot m_{t}=0$$
$$\iff$$
\begin{equation}
    m_{t}=\frac{1}{\tau_{t}}\left(r_{t}^{2}(p_{t}-c_{t})-r_{t}\cdot \tau_{t}\right)=\frac{r_{t}^{2}\left(p_{t}-c_{t}\right)}{\tau_{t}}-r_{t}
\end{equation}

This equation gives us the optimal level of manufacturing investment for the firm. We can see that it has a negative relationship with $\tau$ the pollution tax, which seems logical : the bigger the pollution tax, the more incentives the firm has to invest in R\&D instead of simply manufacturing goods with high pollution level. 

From that, we can identify the optimal level of research $r$ for the firm :

$$B_{t}=r_{t}+m_{t}$$
$$\iff$$
$$B_{t}=r_{t}+\frac{r_{t}^{2}\left(p_{t}-c_{t}\right)}{\tau_{t}}-r_{t}$$
$$\iff$$
$$B_{t}=\frac{r_{t}^{2}\left(p_{t}-c_{t}\right)}{\tau_{t}}$$
$$\iff$$
\begin{equation}
    r_{t}=\sqrt{\frac{B_{t}\cdot \tau_{t}}{p_{t}-c_{t}}}
\end{equation}


Thus, the maximum profit of the firm is : 

$$\pi_{t}= \frac{m_{t}}{c_{t}}\cdot\left(p_{t}-c_{t}\right)-\tau_{t}\cdot \frac{m_{t}}{c_{t}}\cdot \frac{1}{r_{t}}$$
$$\iff$$
$$\pi_{t}=\frac{m_{t}}{c_{t}}\cdot\left(p_{t}-c_{t}-\frac{\tau_{t}}{r_{t}}\right)$$
$$\iff$$
$$\pi_{t}=\frac{m_{t}}{c_{t}}\cdot\left(p_{t}-c_{t}-\frac{\tau_{t}}{\sqrt{\frac{B_{t}\cdot \tau_{t}}{p_{t}-c_{t}}}}\right)$$
$$\iff$$
$$\pi_{t}=\frac{m_{t}}{c_{t}}\cdot\left(p_{t}-c_{t}-\sqrt{\frac{\tau_{t}(p_{t}-c_{t})}{B_{t}}}\right)$$
$$\iff$$
$$\pi_{t}=\frac{B_{t}-\sqrt{\frac{B_{t}\cdot \tau_{t}}{p_{t}-c_{t}}}}{c_{t}}\cdot\left(p_{t}-c_{t}-\sqrt{\frac{\tau_{t}(p_{t}-c_{t})}{B_{t}}}\right)$$
$$\iff$$
$$\pi_{t}=\frac{1}{c_{t}}\cdot\sqrt{B_{t}}\left(\sqrt{B_{t}}-\sqrt{\frac{\tau_{t}}{p_{t}-c_{t}}}\right)\cdot\left(\frac{\sqrt{B_{t}}(p_{t}-c_{t})-\sqrt{\tau(p_{t}-c_{t})}}{\sqrt{B_{t}}}\right)$$
$$\iff$$
$$\pi_{t}=\frac{1}{c_{t}}\cdot\left(\sqrt{B_{t}}-\sqrt{\frac{\tau_{t}}{p_{t}-c_{t}}}\right)\cdot\left(\sqrt{B_{t}}(p_{t}-c_{t})-\sqrt{\tau(p_{t}-c_{t})}\right)$$
$$\iff$$
$$\pi_{t}=\frac{1}{c_{t}}\cdot\left(\frac{\sqrt{B_{t}\cdot (p_{t}-c_{t})}-\sqrt{\tau_{t}}}{\sqrt{p_{t}-c_{t}}}\right)\cdot\left(\sqrt{B_{t}}(p_{t}-c_{t})-\sqrt{\tau(p_{t}-c_{t})}\right)$$
$$\iff$$
$$\pi_{t}=\frac{1}{c_{t}}\cdot\left({\sqrt{B_{t}\cdot (p_{t}-c_{t})}-\sqrt{\tau_{t}}}\right)\cdot\left(\sqrt{B_{t}}(\sqrt{p_{t}-c_{t}})-\sqrt{\tau}\right)$$
$$\iff$$
$$\pi_{t}=\frac{1}{c_{t}}\cdot\left({\sqrt{B_{t}\cdot (p_{t}-c_{t})}-\sqrt{\tau_{t}}}\right)^{2}$$

\subsection{Government problem}

In this simplified model, we say that the Economic Activity $\text{EA}_{t}$ is equivalent to the profit of the firm. Also, we have the following maximisation program : 

$$\max_{\tau_{t},g_{t}}{u_{G,t}=\pi_{F,t}^\alpha\cdot\left(D(X_{t})\right)^{\alpha-1}}$$
subject to : 
\begin{equation*}
    \begin{cases}
        X_t=\rho\cdot X_{t-1}+q_{t}\cdot g_{F,t}+\varepsilon_{t} \\
        X_0=0
    \end{cases}
\end{equation*}

Thus writing all the terms we have : 

$$u_{G,t}=2$$

\end{document}