\documentclass{article}
\usepackage{amsmath}
\usepackage{amssymb}
\usepackage{bm}
\usepackage[dvipsnames]{xcolor}
\usepackage{hyperref}

\title{Microeconomics Project}
\author{Chambon Lionel \\ Gugelmo Cavalheiro Dias Paulo \\ Pavlovich Karina \\ Soufflet Morgane}
\date{\today}
\hypersetup{
    colorlinks,
    citecolor=black,
    filecolor=black,
    linkcolor=black,
    urlcolor=black
}

\begin{document}

\maketitle

\begin{abstract}
    
\end{abstract}
\pagebreak
\tableofcontents
\pagebreak
\section{Baseline model}

This model does not use strategic interactions, and takes place within a perfect information framework. 

\section{Players}

This model has two agents : the Firm and the Governement. 

\subsection{Firm}

\subsubsection{Technology of the firm}

The firm is a monopoly, that has two possible inputs : R\&D investment ($r$) to diminish its pollution per unit and manufacturing investment ($m$) to produce a quantity $q$ of its single good. The firm is also taxed on the pollution level they emit by a tax $\tau$ implemented by the government.

\begin{equation}
    \begin{cases}
        g_{t}=\frac{1}{r_{t}} \\
        m_{t} = q_{t}\cdot c_{t}
    \end{cases}
\end{equation}

Where : 
\begin{itemize}
    \item $g_{t}$ is the pollution generated by the production of one unit of good by the firm at time $t$
    \item $r_{t}$ is the R\&D investment done by the firm to reduce its unit pollution level $g_{t}$
    \item $m_{t}$ is the manufacturing investment of the firm, that is needed to produce a certain quantity of good $q_{t}$
    \item $q_{t}$ is the quantity of good produced by the firm at time $t$
    \item $c_{t}$ is the cost of manufacturing one good by the firm at time $t$
\end{itemize}

\subsubsection{Profit of the firm}

Given those technology constraints, the firm maximises its profit such that : 

\begin{equation}
    \begin{split}
        &\max_{r_{t},m_{t}}{\pi_{F,t}=q_{t}\cdot\left(p_{t}-c_{t}\right)-\tau_{t}\cdot q_{t}\cdot g_{t}} \\
        &\text{subject to : } B_{t} \leq r_{t} + m_{t}
    \end{split}
\end{equation}

Where : 
\begin{itemize}
    \item $q_{t}$ is the quantity of good produced and sold at time $t$
    \item $p_{t}$ is the price at which one unit of good is sold
    \item $c_{t}$ is the cost of production of unit of good
    \item $\tau_{t}\in\left(0,1\right)$ the share at which the one unit of good is taxed, depending on the level of pollution produced per unit, at time $_{t}$
    \item $g_{t}\in\mathbb{R}_{+}$ is the level of pollution created by the production of one unit by the firm at time $t$
    \item $B_{t}$ is the budget of the firm at time $t$
\end{itemize}

\subsection{Government}

The goal of the government is to maximise the social welfare function. The social welfare function here is defined as a utility function taking into account economic activities and the risk of environmental disaster. We define it as : 

\begin{equation}
    u_{G,t}=\text{EA}_{t}^{\alpha}\cdot\text{EQ}_{t}^{1-\alpha}
\end{equation}

Where : 
\begin{itemize}
    \item $\text{EA}_{t}$ is the Economic Activity of society at time $t$
    \item $EQ_{t}$ is the Environmental Quality of the world at time $t$
    \item $\alpha \in \left(0,1\right)$ is the relative importance of Economic Activity compared to the Environmental Quality of the World.
\end{itemize}

This function has two main properties : 

\begin{enumerate}
    \item It has positive but decreasing returns for both Economic Activity and Environmental Quality : 

$$\frac{\partial u_{G,t}}{\partial \text{EA}_{t}}=\alpha\cdot\text{EA}_{t}^{\alpha-1}\cdot\text{EQ}_{t}^{1-\alpha}>0$$
$$\frac{\partial u_{G,t}}{\partial \text{EQ}_{t}}=(1-\alpha)\cdot\text{EA}_{t}^{\alpha}\cdot\text{EQ}_{t}^{-\alpha}>0$$

$$\frac{\partial^2 u_{G,t}}{\partial^2 \text{EA}_{t}}=\alpha\cdot(\alpha-1)\cdot\text{EA}_{t}^{\alpha-2}\cdot\text{EQ}_{t}^{1-\alpha}<0$$
$$\frac{\partial^2 u_{G,t}}{\partial^2 \text{EQ}_{t}}=(1-\alpha)\cdot(-\alpha)\cdot\text{EA}_{t}^{\alpha-1}\cdot\text{EQ}_{t}^{-\alpha-1}<0$$

    \item Both Economic Activity and Environmental Quality are complementary goods, as the cross-derivative is positive :
    
$$\frac{\partial^2 u_{G,t}}{\partial \text{EA}_{t} \partial \text{EQ}_{t}}=\alpha\cdot(1-\alpha)\cdot\text{EA}_{t}^{\alpha-1}\cdot\text{EQ}_{t}^{-\alpha}>0$$
\end{enumerate}

\section{Model solution}

\subsection{Firm's problem}

In this simplified model, we first solve the Firm's problem for one period : 

\begin{equation*}
    \begin{split}
    & \max_{r_{t},m_{t}}{\pi_{F,t}=q_{t}\cdot\left(p_{t}-c_{t}\right)-\tau_{t}\cdot q_{t}\cdot g_{t}} \\ 
    & \text{subject to : } B_{t} \leq r_{t} + m_{t}
    \end{split}
\end{equation*}

We can thus write the Lagrangien such that : 

\begin{equation*}
    \mathcal{L} = q_{t}\cdot\left(p_{t}-c_{t}\right)-\tau_{t}\cdot q_{t}\cdot g_{t}+\lambda\left(B-r_{t}-m_{t}\right)
\end{equation*}

And if we rewrite it with the extended expression of $q_{t}$ and $m_{t}$, we get : 

\begin{equation*}
    \mathcal{L} = \frac{m_{t}}{c_{t}}\cdot\left(p_{t}-c_{t}\right)-\tau_{t}\cdot \frac{m_{t}}{c_{t}}\cdot \frac{1}{r_{t}}+\lambda\left(B-r_{t}-m_{t}\right)
\end{equation*}

We solve the First Order Conditions.

If we differentiate the Lagrangien with respect to the R\&D investment $r_{t}$ :

$$\frac{\partial \mathcal{L}}{\partial r_{t}}=0$$
$$\iff$$
$$\frac{\tau_{t}\cdot m_{t}}{c_{t}\cdot r^{2}_{t}}-\lambda=0$$
$$\iff$$
$$\lambda=\frac{\tau_{t}\cdot m_{t}}{c_{t}\cdot r^{2}_{t}}$$

If we differentiate the Lagrangien with respect to the manufacturing investment $m_{t}$ :

$$\frac{\partial \mathcal{L}}{\partial m_{t}}=0$$
$$\iff$$
$$\frac{p_{t}-c_{t}}{c_{t}}-\frac{\tau_{t}}{c_{t}\cdot r_{t}}-\lambda=0$$
$$\iff$$
$$\frac{p_{t}-c_{t}}{c_{t}}-\frac{\tau_{t}}{c_{t}\cdot r_{t}}-\frac{\tau_{t}\cdot m_{t}}{c_{t}\cdot r^{2}_{t}}=0$$
$$\iff$$
$$r^{2}_{t}\cdot(p_{t}-c_{t})-r_{t}\cdot \tau_{t}-\tau_{t}\cdot m_{t}=0$$
$$\iff$$
\begin{equation}
    m_{t}=\frac{1}{\tau_{t}}\left(r_{t}^{2}(p_{t}-c_{t})-r_{t}\cdot \tau_{t}\right)=\frac{r_{t}^{2}\left(p_{t}-c_{t}\right)}{\tau_{t}}-r_{t}
\end{equation}

This equation gives us the optimal level of manufacturing investment for the firm. We can see that it has a negative relationship with $\tau$ the pollution tax, which seems logical : the bigger the pollution tax, the more incentives the firm has to invest in R\&D instead of simply manufacturing goods with high pollution level. 

From that, we can identify the optimal level of research $r$ for the firm :

$$B_{t}=r_{t}+m_{t}$$
$$\iff$$
$$B_{t}=r_{t}+\frac{r_{t}^{2}\left(p_{t}-c_{t}\right)}{\tau_{t}}-r_{t}$$
$$\iff$$
$$B_{t}=\frac{r_{t}^{2}\left(p_{t}-c_{t}\right)}{\tau_{t}}$$
$$\iff$$
\begin{equation}
    r_{t}=\sqrt{\frac{B_{t}\cdot \tau_{t}}{p_{t}-c_{t}}}
\end{equation}


Thus, the maximum profit of the firm is\footnotemark[1] : 
\footnotetext[1]{\hyperref[sec:proof1]{You can find the proof of this result here.}}
\begin{equation}
    \pi_{t}=\frac{1}{c_{t}}\cdot\left({\sqrt{B_{t}\cdot (p_{t}-c_{t})}-\sqrt{\tau_{t}}}\right)^{2}
\end{equation}

\subsection{Government problem}

In this simplified model, we say that the Economic Activity $\text{EA}_{t}$ is equivalent to the profit of the firm.

The environment quality $EQ_{t}$ at time $t$ depends of the pollution in nature at time $t$. We can define in a simple, one period context : 

\begin{equation}
    EQ_{t}\equiv e^{-q_{t}\cdot g_{t}}
\end{equation}

Where :
\begin{itemize}
    \item $q_{t}\cdot g_{t}$ is the quantity of pollution produced at time $t$
\end{itemize}

Note here that we don't take into account past pollution in this simplified model. This will be taken into account in the next sections. 

Thus, the maximisation program of the government is : 

\begin{equation*}
    \max_{\tau_{t}}{u_{G,t}=\pi_{t}^{\alpha}\cdot e^{-q_{t}\cdot g_{t}\cdot(1-\alpha)}}
\end{equation*}

Now, in this simple framework, the government knows the allocation decisions of the firm, and more specifically knows that the values of $r_{t}$ and $m_{t}$. We can thus rewrite the maximisation program of the government as : 

$$
    \max_{\tau_{t}}{u_{G,t}=\pi_{t}^{\alpha}\cdot {e^{-q_{t}\cdot g_{t}}}^{(1-\alpha)}}
$$
$$\iff$$
$$\max_{\tau_{t}}{\left(\frac{1}{c_{t}}\cdot\left({\sqrt{B_{t}\cdot (p_{t}-c_{t})}-\sqrt{\tau_{t}}}\right)^{2}\right)^{\alpha}\cdot {e^{-q_{t}\cdot g_{t}}}^{(1-\alpha)}}$$

Now : 
$$q_{t}\cdot g_{t}=\frac{m_{t}}{c_{t}}\cdot\frac{1}{r_{t}}=\frac{1}{c_{t}}\left(B_{t}-\sqrt{\frac{B_{t}\cdot \tau_{t}}{p_{t}-c_{t}}}\right)\left(\frac{\sqrt{p_{t}-c_{t}}}{\sqrt{B_{t}\cdot \tau_{t}}}\right)$$
$$\iff$$
$$q_{t}\cdot g_{t}=\frac{1}{c_{t}}\cdot \left(\frac{\sqrt{B_{t}\cdot(p_{t}-c_{t})}}{\sqrt{\tau_{t}}}-1\right)=\frac{1}{c_{t}}\cdot \frac{\sqrt{B_{t}\cdot(p_{t}-c_{t})}-\sqrt{\tau_{t}}}{\sqrt{\tau_{t}}}$$

Thus the objective function of the government is : 
$${u_{G,t}=\left(\frac{1}{c_{t}}\cdot\left({\sqrt{B_{t}\cdot (p_{t}-c_{t})}-\sqrt{\tau_{t}}}\right)^{2}\right)^{\alpha}\cdot {e^{-\frac{1}{c_{t}}\cdot \frac{\sqrt{B_{t}\cdot(p_{t}-c_{t})}-\sqrt{\tau_{t}}}{\sqrt{\tau_{t}}}}}^{(1-\alpha)}}$$
$$\iff$$
$${u_{G,t}=\frac{1}{c_{t}^\alpha}\cdot\left({\sqrt{B_{t}\cdot (p_{t}-c_{t})}-\sqrt{\tau_{t}}}\right)^{2\cdot\alpha}\cdot {e^{-(1-\alpha)\cdot\frac{1}{c_{t}}\cdot \frac{\sqrt{B_{t}\cdot(p_{t}-c_{t})}-\sqrt{\tau_{t}}}{\sqrt{\tau_{t}}}}}}$$

\begin{equation}
    u_{G,t}=\frac{e^{\frac{1}{c_{t}}\cdot\left(1-\alpha\right)}\cdot\left(\sqrt{B(p_{t}-c_{t})}-\sqrt{\tau_{t}}\right)^{2\alpha}}{c^{\alpha}\cdot e^{\frac{1}{c_{t}}\cdot(1-\alpha)\cdot\frac{\sqrt{B_{t}(p_{t}-c_{t})}}{\sqrt{\tau_{t}}}}}
\end{equation}

To maximises this utility function, we set the derivative of it with respect to $\tau_{t}$ to zero, and we solve for $\tau_{t}$, and the pollution tax $\tau$ that maximises the objective function of the government is such that\footnotemark[2] : 
\footnotetext[2]{\hyperref[sec:proof2]{You can find the proof of this result here.}}
\begin{equation}
    \tau_{t}^{*}=\frac{B_{t}\left(p_{t}-c_{t}\right)(1-\alpha)\left[{\alpha \cdot 4\cdot c_{t}+1-\alpha}-\sqrt{(1-\alpha)^{2}+\alpha\cdot8\cdot c_{t}(1-\alpha)}\right]} {\alpha^{2} \cdot 8\cdot c_{t}^{2}}
\end{equation}

After analysis of this term, it appears that : 
\begin{itemize}
    \item It has a negative relationship with the manufacturing cost $c_{t}$
    \item It has a negative relationship with the preference of the governement for economic activity $\alpha$
    \item It has a positive relationship with the budget of the firm cost $B_{t}$
    \item It has a positive relationship with the selling price $p_{t}$
\end{itemize}

\subsection{Interpretations and extensions}

The baseline model happens in only one period, in a simultaneous way, with perfect imperfection. In reality, information is imperfect, decisions of the government and of the firm are not simultaneous, and pollution is persistent. The following extensions will try to develop a framework in which those elements are studied.

\pagebreak
\section{Appendix}

\subsection{Maximum Profit}
\label{sec:proof1}
$$\pi_{t}= \frac{m_{t}}{c_{t}}\cdot\left(p_{t}-c_{t}\right)-\tau_{t}\cdot \frac{m_{t}}{c_{t}}\cdot \frac{1}{r_{t}}$$
$$\iff$$
$$\pi_{t}=\frac{m_{t}}{c_{t}}\cdot\left(p_{t}-c_{t}-\frac{\tau_{t}}{r_{t}}\right)$$
$$\iff$$
$$\pi_{t}=\frac{m_{t}}{c_{t}}\cdot\left(p_{t}-c_{t}-\frac{\tau_{t}}{\sqrt{\frac{B_{t}\cdot \tau_{t}}{p_{t}-c_{t}}}}\right)$$
$$\iff$$
$$\pi_{t}=\frac{m_{t}}{c_{t}}\cdot\left(p_{t}-c_{t}-\sqrt{\frac{\tau_{t}(p_{t}-c_{t})}{B_{t}}}\right)$$
$$\iff$$
$$\pi_{t}=\frac{B_{t}-\sqrt{\frac{B_{t}\cdot \tau_{t}}{p_{t}-c_{t}}}}{c_{t}}\cdot\left(p_{t}-c_{t}-\sqrt{\frac{\tau_{t}(p_{t}-c_{t})}{B_{t}}}\right)$$
$$\iff$$
$$\pi_{t}=\frac{1}{c_{t}}\cdot\sqrt{B_{t}}\left(\sqrt{B_{t}}-\sqrt{\frac{\tau_{t}}{p_{t}-c_{t}}}\right)\cdot\left(\frac{\sqrt{B_{t}}(p_{t}-c_{t})-\sqrt{\tau(p_{t}-c_{t})}}{\sqrt{B_{t}}}\right)$$
$$\iff$$
$$\pi_{t}=\frac{1}{c_{t}}\cdot\left(\sqrt{B_{t}}-\sqrt{\frac{\tau_{t}}{p_{t}-c_{t}}}\right)\cdot\left(\sqrt{B_{t}}(p_{t}-c_{t})-\sqrt{\tau(p_{t}-c_{t})}\right)$$
$$\iff$$
$$\pi_{t}=\frac{1}{c_{t}}\cdot\left(\frac{\sqrt{B_{t}\cdot (p_{t}-c_{t})}-\sqrt{\tau_{t}}}{\sqrt{p_{t}-c_{t}}}\right)\cdot\left(\sqrt{B_{t}}(p_{t}-c_{t})-\sqrt{\tau(p_{t}-c_{t})}\right)$$
$$\iff$$
$$\pi_{t}=\frac{1}{c_{t}}\cdot\left({\sqrt{B_{t}\cdot (p_{t}-c_{t})}-\sqrt{\tau_{t}}}\right)\cdot\left(\sqrt{B_{t}}(\sqrt{p_{t}-c_{t}})-\sqrt{\tau}\right)$$
$$\iff$$
\begin{equation}\tag{6}
    \pi_{t}=\frac{1}{c_{t}}\cdot\left({\sqrt{B_{t}\cdot (p_{t}-c_{t})}-\sqrt{\tau_{t}}}\right)^{2}
\end{equation}

\subsection{Optimal pollution tax}
\label{sec:proof2}

To find the optimla pollution tax, we set the derivative of the social utility function to zero : 

$$u_{G,t}=\frac{e^{\frac{1}{c_{t}}\cdot\left(1-\alpha\right)}\cdot\left(\sqrt{B(p_{t}-c_{t})}-\sqrt{\tau_{t}}\right)^{2\alpha}}{c^{\alpha}\cdot e^{\frac{1}{c_{t}}\cdot(1-\alpha)\cdot\frac{\sqrt{B_{t}(p_{t}-c_{t})}}{\sqrt{\tau_{t}}}}}$$

We see that this function is of form $f(\tau)=\frac{u(\tau)}{v(\tau)}$, thus its derivative is of form $f'(\tau)=\frac{u'(\tau)v(\tau)-v'(\tau)u(\tau)}{v^2(x)}$. To have its derivative equal to zero, we thus only need $u'(\tau)v(\tau)-v'(\tau)u(\tau)=0$, with :

\begin{equation*}
    \begin{cases}
        u(\tau)= e^{\frac{1}{c_{t}}\cdot\left(1-\alpha\right)}\cdot\left(\sqrt{B(p_{t}-c_{t})}-\sqrt{\tau_{t}}\right)^{2\alpha} \\
        v(\tau)= c^{\alpha}\cdot e^{\frac{1}{c_{t}}\cdot(1-\alpha)\cdot\frac{\sqrt{B_{t}(p_{t}-c_{t})}}{\sqrt{\tau_{t}}}}
    \end{cases}
\end{equation*}
$$\iff$$
\begin{equation*}
    \begin{cases}
        \frac{\partial u(\tau)}{\partial \tau}= e^{\frac{1}{c_{t}}\cdot\left(1-\alpha\right)}\cdot \tau^{-\frac{1}{2}}\cdot \left(-\frac{1}{2}\right)\cdot 2\alpha \left(\sqrt{B(p_{t}-c_{t})}-\sqrt{\tau_{t}}\right)^{2\alpha-1} \\
        \frac{\partial v(\tau)}{\partial \tau}= c_{t}^{\alpha}\cdot e^{\frac{1}{c_{t}}\cdot(1-\alpha)\cdot\frac{\sqrt{B_{t}(p_{t}-c_{t})}}{\sqrt{\tau_{t}}}}\cdot \frac{1}{c_{t}}(1-\alpha)\cdot\sqrt{B_{t}(p_{t}-c_{t})}\cdot\left(-\frac{1}{2}\right)\cdot\tau^{-\frac{3}{2}}
    \end{cases}
\end{equation*}
$$\iff$$
\begin{equation*}
    \begin{cases}
        \frac{\partial u(\tau)}{\partial \tau}= -\alpha\cdot e^{\frac{1}{c_{t}}\cdot\left(1-\alpha\right)}\cdot \tau^{-\frac{1}{2}}\cdot \left(\sqrt{B(p_{t}-c_{t})}-\sqrt{\tau_{t}}\right)^{2\alpha-1} \\
        \frac{\partial v(\tau)}{\partial \tau}= -\frac{c_{t}^{\alpha-1}}{2}\cdot e^{\frac{1}{c_{t}}\cdot(1-\alpha)\cdot\frac{\sqrt{B_{t}(p_{t}-c_{t})}}{\sqrt{\tau_{t}}}}\cdot (1-\alpha)\cdot\sqrt{B_{t}(p_{t}-c_{t})}\cdot\tau^{-\frac{3}{2}}
    \end{cases}
\end{equation*}

Thus we can compute : 

\begin{equation*}
    u'(\tau)v(\tau)-v'(\tau)u(\tau)=0
\end{equation*}
$$\iff$$
\begin{equation*}
    \begin{split}
    & -\alpha\cdot e^{\frac{1}{c_{t}}\cdot\left(1-\alpha\right)}\cdot \tau^{-\frac{1}{2}}\cdot \left(\sqrt{B(p_{t}-c_{t})}-\sqrt{\tau_{t}}\right)^{2\alpha-1} \cdot  c^{\alpha}\cdot e^{\frac{1}{c_{t}}\cdot(1-\alpha)\cdot\frac{\sqrt{B_{t}(p_{t}-c_{t})}}{\sqrt{\tau_{t}}}} \\
    & +\frac{c_{t}^{\alpha-1}}{2}\cdot e^{\frac{1}{c_{t}}\cdot(1-\alpha)\cdot\frac{\sqrt{B_{t}(p_{t}-c_{t})}}{\sqrt{\tau_{t}}}}\cdot (1-\alpha)\cdot\sqrt{B_{t}(p_{t}-c_{t})}\cdot\tau^{-\frac{3}{2}}\cdot e^{\frac{1}{c_{t}}\cdot\left(1-\alpha\right)}\cdot\left(\sqrt{B(p_{t}-c_{t})}-\sqrt{\tau_{t}}\right)^{2\alpha}=0
    \end{split}
\end{equation*}
$$\iff$$
\begin{equation*}
    \begin{split}
    & \alpha\cdot e^{\frac{1}{c_{t}}\cdot\left(1-\alpha\right)}\cdot \tau^{-\frac{1}{2}}\cdot \left(\sqrt{B(p_{t}-c_{t})}-\sqrt{\tau_{t}}\right)^{2\alpha-1} \cdot  c^{\alpha} = \\
    & \frac{c_{t}^{\alpha-1}}{2}\cdot (1-\alpha)\cdot\sqrt{B_{t}(p_{t}-c_{t})}\cdot\tau^{-\frac{3}{2}}\cdot e^{\frac{1}{c_{t}}\cdot\left(1-\alpha\right)}\cdot\left(\sqrt{B(p_{t}-c_{t})}-\sqrt{\tau_{t}}\right)^{2\alpha}
    \end{split}
\end{equation*}
$$\iff$$
\begin{equation*}
    \begin{split}
    & \alpha\cdot e^{\frac{1}{c_{t}}\cdot\left(1-\alpha\right)}\cdot \tau^{-\frac{1}{2}}\cdot \left(\sqrt{B(p_{t}-c_{t})}-\sqrt{\tau_{t}}\right)^{-1} \cdot  c^{\alpha} = \\
    & \frac{c_{t}^{\alpha-1}}{2}\cdot (1-\alpha)\cdot\sqrt{B_{t}(p_{t}-c_{t})}\cdot\tau^{-\frac{3}{2}}\cdot e^{\frac{1}{c_{t}}\cdot\left(1-\alpha\right)}
    \end{split}
\end{equation*}
$$\iff$$
\begin{equation*}
    \begin{split}
    & \alpha\cdot \tau^{-\frac{1}{2}}\cdot \left(\sqrt{B(p_{t}-c_{t})}-\sqrt{\tau_{t}}\right)^{-1} \cdot  c^{\alpha} = \\
    & \frac{c_{t}^{\alpha-1}}{2}\cdot (1-\alpha)\cdot\sqrt{B_{t}(p_{t}-c_{t})}\cdot\tau^{-\frac{3}{2}} 
    \end{split}
\end{equation*}
$$\iff$$
\begin{equation*}
    \alpha\cdot \tau^{-\frac{1}{2}}\cdot \left(\sqrt{B(p_{t}-c_{t})}-\sqrt{\tau_{t}}\right)^{-1} = \frac{1}{2\cdot c_{t}}\cdot (1-\alpha)\cdot\sqrt{B_{t}(p_{t}-c_{t})}\cdot\tau^{-\frac{3}{2}} 
\end{equation*}
$$\iff$$
\begin{equation*}
    \alpha\cdot \left(\sqrt{B(p_{t}-c_{t})}-\sqrt{\tau_{t}}\right)^{-1} = \frac{1}{2\cdot c_{t}}\cdot (1-\alpha)\cdot\sqrt{B_{t}(p_{t}-c_{t})}\cdot\tau_{t}^{-1} 
\end{equation*}
$$\iff$$
\begin{equation*}
    \alpha = \left(\sqrt{B(p_{t}-c_{t})}-\sqrt{\tau_{t}}\right)\cdot \frac{1}{2\cdot c_{t}}\cdot (1-\alpha)\cdot\sqrt{B_{t}(p_{t}-c_{t})}\cdot\tau_{t}^{-1} 
\end{equation*}
$$\iff$$
\begin{equation*}
    \frac{\alpha \cdot 2\cdot c_{t}}{(1-\alpha)\cdot\sqrt{B_{t}(p_{t}-c_{t})}}= \frac{\left(\sqrt{B(p_{t}-c_{t})}-\sqrt{\tau_{t}}\right)}{\tau_{t}} 
\end{equation*}
$$\iff$$
\begin{equation*}
    \tau_{t}\cdot\frac{\alpha \cdot 2\cdot c_{t}}{(1-\alpha)\cdot\sqrt{B_{t}(p_{t}-c_{t})}}= \sqrt{B(p_{t}-c_{t})}-\sqrt{\tau_{t}} 
\end{equation*}
$$\iff$$
\begin{equation*}
    \tau_{t}\cdot\frac{\alpha \cdot 2\cdot c_{t}}{(1-\alpha)}+\sqrt{\tau_{t}}\cdot\sqrt{B_{t}(p_{t}-c_{t})}-B_{t}(p_{t}-c_{t})=0
\end{equation*}
$$\iff$$
\begin{equation*}
    \tau_{t}\cdot\frac{\alpha \cdot 2\cdot c_{t}}{(1-\alpha)}-B_{t}(p_{t}-c_{t})=-\sqrt{\tau_{t}}\cdot\sqrt{B_{t}(p_{t}-c_{t})}
\end{equation*}
$$\iff$$
\begin{equation*}
    \tau_{t}^{2}\cdot\left(\frac{\alpha \cdot 2\cdot c_{t}}{(1-\alpha)}\right)^{2}-2\cdot\tau_{t}\cdot B_{t}\left(p_{t}-c_{t}\right)\cdot \frac{\alpha \cdot 2\cdot c_{t}}{(1-\alpha)}+\left(B_{t}(p_{t}-c_{t})\right)^{2}={\tau_{t}}\cdot{B_{t}(p_{t}-c_{t})}
\end{equation*}
$$\iff$$
\begin{equation*}
    \tau_{t}^{2}\cdot\left(\frac{\alpha \cdot 2\cdot c_{t}}{(1-\alpha)}\right)^{2}+\tau_{t}\cdot(-2)\cdot B_{t}\left(p_{t}-c_{t}\right)\cdot \frac{\alpha \cdot 2\cdot c_{t}}{(1-\alpha)}-{\tau_{t}}\cdot{B_{t}(p_{t}-c_{t})}+\left(B_{t}(p_{t}-c_{t})\right)^{2}=0
\end{equation*}
$$\iff$$
\begin{equation*}
    \tau_{t}^{2}\cdot\left(\frac{\alpha \cdot 2\cdot c_{t}}{(1-\alpha)}\right)^{2}+\tau_{t}\cdot B_{t}\left(p_{t}-c_{t}\right)\left((-2)\cdot \frac{\alpha \cdot 2\cdot c_{t}}{(1-\alpha)}-1\right)+\left(B_{t}(p_{t}-c_{t})\right)^{2}=0
\end{equation*}
$$\iff$$
\begin{equation*}
    \tau_{t}^{2}\cdot\left(\frac{\alpha \cdot 2\cdot c_{t}}{(1-\alpha)}\right)^{2}+\tau_{t}\cdot \left(-B_{t}\right)\left(p_{t}-c_{t}\right)\left(2\cdot \frac{\alpha \cdot 2\cdot c_{t}}{(1-\alpha)}+1\right)+\left(B_{t}(p_{t}-c_{t})\right)^{2}=0
\end{equation*}

This is a standard quadratic equation $\tau_{t}^{2}\cdot a+\tau_{t}\cdot b + c =0$ that accepts two solutions : 

$$\tau^{*}=\frac{-b\pm\sqrt{b^2-4\cdot a\cdot c}}{2\cdot a}$$

With : 

\begin{itemize}
    \item $a = \left(\frac{\alpha \cdot 2\cdot c_{t}}{1-\alpha}\right)^{2}$
    \item $b = \left(-B_{t}\right)\left(p_{t}-c_{t}\right)\left(2\cdot \frac{\alpha \cdot 2\cdot c_{t}}{(1-\alpha)}+1\right)$
    \item $c = \left(B_{t}(p_{t}-c_{t})\right)^{2}$
\end{itemize}

The solution $\tau^{*}=\frac{-b+\sqrt{b^2+4\cdot a\cdot c}}{2\cdot a}$ does not make sense in the context of this problem, so we have the optimal tax : 

\begin{equation*}
    \tau_{t}^{*}=\frac{-b-\sqrt{b^2-4\cdot a\cdot c}}{2\cdot a}
\end{equation*}
$$\iff$$
\begin{equation*}
    \tau_{t}^{*}=\frac{B_{t}\left(p_{t}-c_{t}\right)\left(2\cdot \frac{\alpha \cdot 2\cdot c_{t}}{(1-\alpha)}+1\right)-\sqrt{\left[\left(-B_{t}\right)\left(p_{t}-c_{t}\right)\left(2\cdot \frac{\alpha \cdot 2\cdot c_{t}}{(1-\alpha)}+1\right)\right]^2-4\cdot \left(\frac{\alpha \cdot 2\cdot c_{t}}{1-\alpha}\right)^{2}\cdot \left(B_{t}(p_{t}-c_{t})\right)^{2}}}{2\cdot \left(\frac{\alpha \cdot 2\cdot c_{t}}{1-\alpha}\right)^{2}}
\end{equation*}
$$\iff$$
\begin{equation*}
    \tau_{t}^{*}=\frac{B_{t}\left(p_{t}-c_{t}\right)\left(2\cdot \frac{\alpha \cdot 2\cdot c_{t}}{(1-\alpha)}+1\right)-\sqrt{\left(B_{t}\left(p_{t}-c_{t}\right)\right)^{2}\left(2\cdot \frac{\alpha \cdot 2\cdot c_{t}}{(1-\alpha)}+1\right)^2-4\cdot \left(\frac{\alpha \cdot 2\cdot c_{t}}{1-\alpha}\right)^{2}\cdot \left(B_{t}(p_{t}-c_{t})\right)^{2}}}{2\cdot \left(\frac{\alpha \cdot 2\cdot c_{t}}{1-\alpha}\right)^{2}}
\end{equation*}
$$\iff$$
\begin{equation*}
    \tau_{t}^{*}=\frac{B_{t}\left(p_{t}-c_{t}\right)\left(2\cdot \frac{\alpha \cdot 2\cdot c_{t}}{(1-\alpha)}+1\right)-\sqrt{\left(B_{t}\left(p_{t}-c_{t}\right)\right)^{2}\left[\left(2\cdot \frac{\alpha \cdot 2\cdot c_{t}}{(1-\alpha)}+1\right)^2-4\cdot \left(\frac{\alpha \cdot 2\cdot c_{t}}{1-\alpha}\right)^{2}\right]}}{2\cdot \left(\frac{\alpha \cdot 2\cdot c_{t}}{1-\alpha}\right)^{2}}
\end{equation*}
$$\iff$$
\begin{equation*}
    \tau_{t}^{*}=\frac{B_{t}\left(p_{t}-c_{t}\right)\left(2\cdot \frac{\alpha \cdot 2\cdot c_{t}}{(1-\alpha)}+1\right)-B_{t}\left(p_{t}-c_{t}\right)\sqrt{\left(2\cdot \frac{\alpha \cdot 2\cdot c_{t}}{(1-\alpha)}+1\right)^2-4\cdot \left(\frac{\alpha \cdot 2\cdot c_{t}}{1-\alpha}\right)^{2}}}{2\cdot \left(\frac{\alpha \cdot 2\cdot c_{t}}{1-\alpha}\right)^{2}}
\end{equation*}
$$\iff$$
\begin{equation*}
    \tau_{t}^{*}=\frac{B_{t}\left(p_{t}-c_{t}\right)\left[\left(2\cdot \frac{\alpha \cdot 2\cdot c_{t}}{1-\alpha}+1\right)-\sqrt{\left(2\cdot \frac{\alpha \cdot 2\cdot c_{t}}{1-\alpha}+1\right)^2-4\cdot \left(\frac{\alpha \cdot 2\cdot c_{t}}{1-\alpha}\right)^{2}}\right]}{2\cdot \left(\frac{\alpha \cdot 2\cdot c_{t}}{1-\alpha}\right)^{2}}
\end{equation*}
$$\iff$$
\begin{equation*}
    \tau_{t}^{*}=\frac{B_{t}\left(p_{t}-c_{t}\right)\left[\left(\frac{\alpha \cdot 4\cdot c_{t}+1-\alpha}{1-\alpha}\right)-\sqrt{\left(\frac{\alpha \cdot 4\cdot c_{t}+1-\alpha}{1-\alpha}\right)^2-\frac{\alpha^{2} \cdot 16\cdot c_{t}^{2}}{(1-\alpha)^{2}}}\right]}{\frac{\alpha^{2} \cdot 8\cdot c_{t}^{2}}{(1-\alpha)^{2}}}
\end{equation*}
$$\iff$$
\begin{equation*}
    \tau_{t}^{*}=\frac{B_{t}\left(p_{t}-c_{t}\right)\left[\left(\frac{\alpha \cdot 4\cdot c_{t}+1-\alpha}{1-\alpha}\right)-\sqrt{\frac{1}{(1-\alpha)^{2}}\left(\left(\alpha \cdot 4\cdot c_{t}+1-\alpha\right)^2-\left(\alpha^{2} \cdot 16\cdot c_{t}^{2}\right)\right)}\right]} {\frac{\alpha^{2} \cdot 8\cdot c_{t}^{2}}{(1-\alpha)^{2}}}
\end{equation*}
$$\iff$$
\begin{equation*}
    \tau_{t}^{*}=\frac{B_{t}\left(p_{t}-c_{t}\right)\left[\left(\frac{\alpha \cdot 4\cdot c_{t}+1-\alpha}{1-\alpha}\right)-\frac{1}{(1-\alpha)}\sqrt{\alpha^{2} \cdot 16\cdot c_{t}^{2}+(1-\alpha)^{2}+2\cdot \alpha\cdot4\cdot c_{t}(1-\alpha)-\alpha^{2} \cdot 16\cdot c_{t}^{2}}\right]} {\frac{\alpha^{2} \cdot 8\cdot c_{t}^{2}}{(1-\alpha)^{2}}}
\end{equation*}
$$\iff$$
\begin{equation*}
    \tau_{t}^{*}=\frac{B_{t}\left(p_{t}-c_{t}\right)(\frac{1}{1-\alpha})\left[{\alpha \cdot 4\cdot c_{t}+1-\alpha}-\sqrt{(1-\alpha)^{2}+2\cdot \alpha\cdot4\cdot c_{t}(1-\alpha)}\right]} {\frac{\alpha^{2} \cdot 8\cdot c_{t}^{2}}{(1-\alpha)^{2}}}
\end{equation*}
$$\iff$$
\begin{equation*}\tag{9}
    \tau_{t}^{*}=\frac{B_{t}\left(p_{t}-c_{t}\right)(1-\alpha)\left[{\alpha \cdot 4\cdot c_{t}+1-\alpha}-\sqrt{(1-\alpha)^{2}+\alpha\cdot8\cdot c_{t}(1-\alpha)}\right]} {\alpha^{2} \cdot 8\cdot c_{t}^{2}}
\end{equation*}

\end{document}